% $Id: pyclips.tex 342 2008-02-22 01:17:23Z Franz $
\documentclass{manual}

% some definitions used troughout the documentation
% $Id: defs.tex 99 2004-07-13 21:01:38Z Franz $

% all references to the PyCLIPS module should use this macro
\newcommand{\pyclips}{\module{PyCLIPS}}


% references to the APG, BPG and Tutorial should use these macros
\newcommand{\clipsapg}{\emph{Clips Reference Guide Vol. II:
Advanced Programming Guide}}
\newcommand{\sclipsapg}{\seetitle{Clips Reference Guide Vol.
II: Advanced Programming Guide}{contains detailed information
about CLIPS API and internal structures}}

\newcommand{\clipsbpg}{\emph{Clips Reference Guide Vol. I:
Basic Programming Guide}}
\newcommand{\sclipsbpg}{\seetitle{Clips Reference Guide Vol.
I: Basic Programming Guide}{the main reference for the CLIPS
language}}

\newcommand{\clipstut}{\emph{Clips User's Guide}}
\newcommand{\sclipstut}{\seetitle{Clips User's Guide}{the
official tutorial for the CLIPS language}}



\title{\pyclips{} Manual}

\author{Francesco Garosi}

% Please at least include a long-lived email address;
% the rest is at your discretion.
\authoraddress{
    E-mail: \email{franz -dot- g -at- tin -dot- it}
}

% do not mess with the 2 lines below, they are written by make dist
\release{1.0}
\date{Feb 22, 2008}

\makeindex

\begin{document}

\maketitle

% This makes the contents accessible from the front page of the HTML.
\ifhtml
\chapter*{Front Matter\label{front}}
\fi

% $Id: copyright.tex 342 2008-02-22 01:17:23Z Franz $
Copyright \copyright{} 2002-2008 Francesco Garosi/JKS.
All rights reserved.

See the end of this document for complete license and permissions
information.


\begin{abstract}

\noindent
This manual documents the high-level interface to the CLIPS system
provided by the \pyclips{} module. This module allows the creation of a
fully-functional CLIPS environment in a Python session, thus providing
access to a well-known expert systems shell programmatically.

\pyclips{} incorporates most of the API bindings documented in the
\clipsapg{} (freely downloadable from the CLIPS web site, see below) and
embeds these bindings in an Object Oriented layer, incorporating most
CLIPS constructs into Python classes. Instances of these classes allow
access to the status and functionality of the corresponding CLIPS objects.

\end{abstract}

\note{This manual is not intended to be a documentation of CLIPS itself.
CLIPS is extensively documented in several manuals, which are available on
the CLIPS website (see below). A comprehensive tutorial for CLIPS can also
be downloaded. A reasonable knowledge of the CLIPS system is necessary
to understand and use this module.}

\begin{seealso}
  \seetitle[http://www.python.org/]
	{Python Language Web Site}{for information on the Python language}
  \seetitle[http://www.ghg.net/clips/CLIPS.html]
	{CLIPS system web site}{for information on the CLIPS system}
\end{seealso}

\tableofcontents

% $Id: intro.tex 346 2008-02-25 00:39:00Z Franz $
\chapter{Introduction\label{introduction}}

\section{Overview\label{pyclips-overview}}

This module aims to embed a fully functional CLIPS engine in Python, and
to give to the developer a more Python-compliant interface to CLIPS
without cutting down on functionalities. In fact CLIPS is compiled into
the module in its entirety, and most API functions are bound to Python
methods. However the direct bindings to the CLIPS library (implemented
as the \module{_clips} submodule) are not described here: each function
is described by an appropriate documentation string, and accessible by
means of the \function{help()} function or through the \program{pydoc}
tool. Each direct binding maps to an API provided function. For a
detailed reference\footnote{The order of parameters is changed sometimes,
in order to allow a more intuitive use of default parameters in the
Python interface: however the meaning of each parameter is described
in the function documentation string, and it should not be difficult
for the programmer to correctly understand the relationship between a
module function and the corresponding CLIPS API.} for these functions
see \clipsapg{}, available for download at the CLIPS website.

\pyclips{} is also capable of generating CLIPS text and binary files:
this allows the user to interact with sessions of the CLIPS system
itself.

An important thing to know, is that \pyclips{} implements CLIPS as a
separated\footnote{This has an impact on the way the module can be used,
as the engine is only set up once when the module is \code{import}ed the
first time.} engine: in the CLIPS module implementation, CLIPS
``lives'' in its own memory space, allocates its own objects. The module
only provides a way to send information and commands to this engine
and to retrieve results from it.


\subsection{Structure\label{pyclips-ov-structure}}

\pyclips{} is organized in a package providing several classes and
top-level functions. Also, the module provides some objects that are
already instanced and give access to some of the CLIPS internal
functions and structures, including debug status and engine I/O.

CLIPS is accessible through these classes and functions, that send
appropriate commands to the underlying engine and retrieve the
available information. Many of the CLIPS classes, constructs and
objects are shadowed by Python classes and objects. However, whereas
\pyclips{} classes provide a comfortable way to create objects that
reference the actual engine objects, there is no one-to-one
mapping between the two memory spaces: for instance, when a Python
object is deleted (via the \keyword{del} command), the corresponding
CLIPS object will still remain alive in the CLIPS memory space. An
appropriate command is necessary to remove the object from the
underlying engine, and this is provided by the module interface.


\subsection{Interactive Usage\label{pyclips-ov-interactive}}

The \pyclips{} package can also be used interactively, since it can
inspect an underlying CLIPS session and give some of the output that
CLIPS usually provides when used as an interactive shell.

A simple interactive session with \pyclips{} follows:

\begin{verbatim}
>>> import clips
>>> clips.Reset()
>>> clips.Assert("(duck)")
<Fact 'f-1': fact object at 0x00DE4AE0>
>>> clips.BuildRule("duck-rule", "(duck)", "(assert (quack))", "the Duck Rule")
<Rule 'duck-rule': defrule object at 0x00DA7E00>
>>> clips.PrintRules()
MAIN:
duck-rule
>>> clips.PrintAgenda()
MAIN:
   0      duck-rule: f-1
For a total of 1 activation.
>>> clips.PrintFacts()
f-0     (initial-fact)
f-1     (duck)
For a total of 2 facts.
>>> clips.Run()
>>> clips.PrintFacts()
f-0     (initial-fact)
f-1     (duck)
f-2     (quack)
For a total of 3 facts.
\end{verbatim}

Users of the CLIPS interactive shell will find the \pyclips{} output
quite familiar. In fact the \function{Print\emph{<object>}()} functions
are provided for interactive use, and retrieve their output directly from
the underlying CLIPS engine I/O subsystem, in order to resemble an
interactive CLIPS session. Other functions are present to retrieve
object names, values and the so called \emph{pretty-print forms} for
programmatic use.


\section{Implementation Structure\label{pyclips-implstructure}}

This section describes the guidelines and considerations that lead
to this CLIPS interface implementation. For the developers which
normally use CLIPS as a development environment or expert systems
shell, the architecture of the \pyclips{} module will look a little
bit different, and in some ways could also seem confusing.

The main topics covered by these sections are the \emph{Implementation
of Constructs as Classes}, the \emph{Implementation of CLIPS I/O
Subsystem}, the \emph{Configuration and Debug Objects}, the
\emph{Coexistence of Global and Environment-Aware Engines} and the
\emph{Conventions used for Naming} which explains the rules that
helped choose the current naming scheme for classes, functions and
objects implemented in \pyclips{}.


\subsection{Implementation of Constructs as Classes\label{pyclips-ov-casc}}

CLIPS users know that this shell offers several constructs to populate
the system memory. These constructs are not described here, since a
detailed explaination of the CLIPS language can be found in the
official CLIPS documentation. These constructs, many of which have
their particular syntax, create ``objects'' (not necessarily in the
sense of OOP\footnote{In fact, the word \emph{object} is used here
to indicate an item that takes part in a program. For instance, one
such object can be a \class{Rule}: it is not a proper OO object,
but something that in imperative languages would act as a control
structure.}, although some of these can be \class{Instance}s of
\class{Class}es) in the subsystem memory.

The choice of implementing most of these constructs as classes gives
to \pyclips{} a more organic structure. Most of the construct classes
share similarities which make the interface structure simpler and the
access to CLIPS objects more systematic.

Most constructs are implemented as \emph{factory functions} which
return instances of Python classes. These Python instances (that
shadow the corresponding CLIPS objects), on their turn, have methods
and properties which operate directly on the objects they map in the
CLIPS subsystem. Methods and properties provide both access and
\emph{send messages} to these objects.

An example of this follows:

\begin{verbatim}
>>> import clips
>>> f0 = clips.Assert("(duck)")
>>> print f0
f-0
>>> print f0.Exists()
True
>>> f0.Retract()
>>> print f0.Exists()
False
\end{verbatim}

In the above example, a fact (\code{(duck)}) is asserted and then
retracted. The assertion is done by means of a module-level function
(\function{Assert()}) and the fact is retracted using a method of the
shadow object (\var{f0}). A verification on the CLIPS \code{fact} object,
using the Python \class{Fact} instance\footnote{It should be clear
that terminology differs semantically from Python system to the CLIPS
system: while OOP, to which terms used in Python are coherent, uses
the words \emph{method}, \emph{instance} and so on with a particular
meaning (to which Python developers are familiar), CLIPS terminology
often differs from OOP, sometimes only slightly but at other times more
substantially. The reader should note that throughout this manual
each term is used -- as far as possible -- with the meaning that it
assumes in its specific environment. In this case, the word
\emph{instance} represents the instance of a Python \keyword{class},
and is not referred to an entity in CLIPS.} \var{f0}'s method
\function{Exists()}, shows that after invoking the \function{Retract()}
method of \var{f0} the \code{fact} object does no longer exists in the
CLIPS subsystem, thus it has been actually retracted.

As stated previously, this does not remove the Python object
(a \class{Fact} instance) from the namespace pertinent to Python itself:
as it can be seen from the code snip shown above, \code{f0} is still a
functional instance, and can be queried about the existence of the
corresponding object in CLIPS.

Objects in the CLIPS operating space can be referenced by more than a
Python object (or even not be referenced at all, if CLIPS creation does
not correspond to a Python assignment), as demonstrated by the following
code:

\begin{verbatim}
>>> clips.Reset()
>>> f1 = clips.Assert("(duck)")
>>> clips.Assert("(quack)")
<Fact 'f-2': fact object at 0x00DE8420>
>>> f1
<Fact 'f-1': fact object at 0x00DE3020>
>>> fl = clips.FactList()
>>> f1b = fl[1]
>>> f1b
<Fact 'f-1': fact object at 0x00E08C40>
\end{verbatim}

Both \var{f1} and \var{f1b} refer to the same object in CLIPS namespace,
but their address is different: in fact they are two different Python
objects (equality test fails) but correspond to the same \code{fact} in
CLIPS.

\begin{seealso}
\sclipsbpg{}
\end{seealso}


\subsection{Implementation of CLIPS I/O Subsystem\label{pyclips-ov-io}}

The CLIPS shell interacts with the user answering to typed in commands
with some informational output. An interactive CLIPS session will show
this:

\begin{verbatim}
CLIPS> (reset)
CLIPS> (watch activations)
CLIPS> (defrule duck-rule "the Duck Rule"
   (duck)
=>
   (assert (quack)))
CLIPS> (assert (duck))
==> Activation 0      duck-rule: f-1
<Fact-1>
CLIPS> (run)
CLIPS> (facts)
f-0     (initial-fact)
f-1     (duck)
f-2     (quack)
For a total of 3 facts.
\end{verbatim}

Each time a \code{fact} is asserted, CLIPS outputs a string containing
its index, and since we decided to show some debug output about
activations, CLIPS produces a line as soon as \code{duck} is asserted,
since \code{duck-rule} would be activated by this. Although in an
interactive session all of the output would go to the terminal, CLIPS
logically considers the ``streams'' for different output types as
separated: in fact, debug output (the one generated by the \code{watch}
command) goes to a special stream called \code{wtrace}. In this special
case, for instance, the debug output can be captured by \pyclips{}
through a special stream-like Python object, which provides a
\function{Read()} function\footnote{Note that \function{Read()} is
capitalized: this is because the stream-like objects do not really act
as ``files'' as many objects which can be read in Python do. So it
becomes impossible to use these \pyclips{} objects where a file-like
object is to be used.}. Comparing the behaviour of two interactive
sessions, the former in the CLIPS subsystem and the latter in Python,
will help to understand the close relationship between CLIPS I/O and
\pyclips{} stream objects. CLIPS will interact with the user as
follows:

\begin{verbatim}
CLIPS> (defrule sayhello
   (hello)
=>
   (printout t "hello, world!" crlf))
CLIPS> (assert (hello))
==> Activation 0      sayhello: f-1
<Fact-1>
CLIPS> (run)
hello, world!
\end{verbatim}

And the Python counterpart follows:

\begin{verbatim}
>>> import clips
>>> clips.DebugConfig.ActivationsWatched = True
>>> r0 = clips.BuildRule("sayhello", "(hello)",
                         '(printout stdout "hello, world!" crlf)')
>>> print r0.PPForm()
(defrule MAIN::sayhello
   (hello)
   =>
   (printout stdout "hello, world!" crlf))

>>> clips.Assert("(hello)")
<Fact 'f-0': fact object at 0x00DE81C0>
>>> t = clips.TraceStream.Read()
>>> print t
==> Activation 0      sayhello: f-0
>>> clips.Run()
>>> t = clips.StdoutStream.Read()
>>> print t
hello, world!
\end{verbatim}

The I/O access objects can be used both in interactive and unattended
sessions. In this case, they can be useful to retrieve periodical
information about CLIPS internal status, since most of the output
provided can be easily interpreted in a programmatic way. Also, there
is one more stream called \var{StdinStream} (which has a
\function{Write()} method) that might be useful to send input to the
CLIPS engine when some ``user interaction'' is required\footnote{This only
happens actually when CLIPS invokes the \code{read} or \code{readline}
functions.}.

There is no way to create other instances of these streams: the
high-level module hides the class used to build these objects. This
is because the I/O streams have to be considered like ``physical
devices'' whose use is reserved to the engine to report trace and
debug information as well as user requested output.

These I/O streams will be described later in the detail, since each
one can be used to report about a specific task.

\note{The streams have to be explicitly read: there is no way to
receive a notification from CLIPS that some output has been written.
In other words, the \pyclips{} engine is not \emph{event driven} in
its interaction with Python.}


\subsection{Configuration and Debug Objects\label{pyclips-ov-cado}}

As well as I/O streams, there are two other objects directly provided
by \pyclips{}. These objects provide access to the CLIPS engine global
configuration. Many aspects of the CLIPS engine, that in the command
line environment would be configured using particular commands, are
accessible via the \var{EngineConfig} (global engine configuration)
object and the \var{DebugConfig} (global debug and trace configuration)
object. For example, we can take the code snip shown above:

\begin{verbatim}
>>> clips.DebugConfig.ActivationsWatched = True
\end{verbatim}
(...)
\begin{verbatim}
>>> t = clips.TraceStream.Read()
>>> print t
==> Activation 0      sayhello: f-0
\end{verbatim}

The \code{clips.DebugConfig.ActivationsWatched = True} line tells to the
underlying subsystem that debug information about \emph{rule activations}
has to be written to the proper stream (the stream dedicated to debug
output in CLIPS is called \code{wtrace} and is accessible in \pyclips{}
through the \var{TraceStream} object).

As it has been said for the I/O streams, these objects cannot be instanced
by the user: access to these objects affects global (or at least
\emph{environmental}, we will see the difference later) configuration,
so it would be of no meaning for the user to create more, possibly
confusing, instances of such objects.


\subsection{Coexistence of Global and Environment-Aware Engines\label{pyclips-ov-env}}

As of version 6.20, CLIPS API offers the possibility to have several
\emph{environments} in which to operate. We can consider environments as
separate engines that only share the operating mode, in other words
``the code''. \pyclips{} also implements environments by means of a special
\class{Environment} class. This class implements all the features
provided by the top level methods and classes. The \class{Environment}
class reimplements all classes provided by \pyclips{}, but -- although
their behaviour is quite similar -- methods of classes provided by
\class{Environment} only affect the CLIPS environment represented by the
\class{Environment} instance itself.

There is normally no need to use environments. However, access to them
is provided for CLIPS ``gurus'' who want to have more than one engine
working at the same time. The end user of \pyclips{} will see no real
difference between a call to a function and its environmental counterpart
(defined as \emph{companion function} in the official CLIPS documentation),
apart from being called as a member function of an \class{Environment}
object.

A simple example will be explanatory:

\begin{verbatim}
>>> clips.Clear()
>>> clips.Reset()
>>> e0 = clips.Environment()
>>> e1 = clips.Environment()
>>> e0.Assert("(duck)")
<Fact 'f-0': fact object at 0x00E7D960>
>>> e1.Assert("(quack)")
<Fact 'f-0': fact object at 0x00E82220>
>>> e0.PrintFacts()
f-0     (duck)
For a total of 1 fact.
>>> e1.PrintFacts()
f-0     (quack)
For a total of 1 fact.
\end{verbatim}


\subsection{Using External Functions in CLIPS\label{pyclips-ov-extfuncs}}

\pyclips{} gives the ability to users to call Python code from within
the CLIPS subsystem. Virtually every function defined in Python can be
called from CLIPS code using the special CLIPS function \code{python-call}.
However, since CLIPS has different basic types than Python, in most cases
it would be useful for modules that implement function to be called in
the CLIPS engine to import the \pyclips{} module themselves, in order to
be aware of the structures that CLIPS uses.

Functions have to be registered in \pyclips{} in order to be available
to the underlying engine, and the registration process can dynamically
occur at any moment.

A simple example follows:

\begin{verbatim}
>>> import clips
>>> def py_square(x):
        return x * x
>>> clips.RegisterPythonFunction(py_square)
>>> print clips.Eval("(python-call py_square 7)")
49
>>> print clips.Eval("(python-call py_square 0.7)")
0.49
\end{verbatim}

A more detailed description of the features provided by \code{python-call}
can be found in the appendices.


\subsection{Conventions Used for Naming\label{pyclips-ov-names}}

In \pyclips{}, the simple convention that is used is that all valuable
content exposed has a name beginning with a capital letter. Names
beginning with a single underscore have normally no meaning for the
\pyclips{} user. Functions, class names and objects use mixed capitals
(as in Java), and \emph{manifest constants} (names used in \emph{lieu}
of explicit values to pass instructions to CLIPS functions or
properties) are all capitalized, as is usual for the C language.

CLIPS users will perhaps be confused because often the constructs in
CLIPS are expressed by keywords containing a \code{def} prefix. The
choice was made in \pyclips{} to drop this prefix in many cases: the use
of this prefix has a strong logic in the CLIPS language, because in this
way the developer knows that a \emph{construct} is used, that is, a
\emph{definition} is made. The keyword used to instance this definition,
both encapsulates the meaning of \code{def}inition itself, and also
the type of construct that is being defined (e.g. \code{def}ine a
\code{rule} is \code{defrule}), thus avoiding making constructs more
difficult by means of two separate keywords. In \pyclips{}, since the
definition happens at class declaration and the instantiation of classes
shadows a construct definition when it has already been performed, it
seemed unnecessary to keep the prefix: in fact, to follow the above
example, it does not seem correct to refer to a rule within the CLIPS
subsystem as a ``\class{Defrule}'' object, hence it is simply referred
to as a \class{Rule}.


\subsection{Pickling Errors\label{pyclips-ov-pickle}}

Python objects cannot be pickled or unpickled. This is because, since
pickling an object would save a reference to a CLIPS entity -- which is
useless across different \pyclips{} sessions -- the unpickling process
would feed the underlying engine in an unpredictable way, or at least
would reference memory locations corresponding to previous CLIPS entities
without the engine having them allocated.

One better way to achieve a similar goal is to use the \function{Save()}
or \function{BSave()} (and related \function{Load()} or \function{BLoad()})
to save the engine\footnote{The mentioned functions are also \emph{members}
of the \class{Environment} class, in which case the \class{Environment}
status is saved.} status in its entirety.

If a single entity is needed, its \emph{pretty-print form} can be used
in most cases to recreate it using the \function{Build()} functions.


\section{Other Usage Modes\label{pyclips-ov-otheruses}}

It is also interesting that, by using some particular functions and the
provided I/O subsystem, even ``pure'' CLIPS programs can be executed by
\pyclips{}, and while the simple output from CLIPS can be read to obtain
feedback, the possibility of inspecting the internal CLIPS subsystem
state remains.

The following example, taken from the CLIPS website\footnote{In fact
the file has been slightly reformatted for typesetting reasons.},
illustrates this: first we take a full CLIPS program, saved as
\file{zebra.clp}, and reported below:

\verbatiminput{zebra.clp}

then we execute all commands (using the \function{BatchStar()} function)
in the current \class{Environment} of an interactive \pyclips{} session:

\begin{verbatim}
>>> clips.BatchStar("zebra.clp")
>>> clips.Reset()
>>> clips.Run()
>>> s = clips.StdoutStream.Read()
>>> print s
There are five houses, each of a different color, inhabited by men of
different nationalities, with different pets, drinks, and cigarettes.

The Englishman lives in the red house.  The Spaniard owns the dog.
The ivory house is immediately to the left of the green house, where
the coffee drinker lives.  The milk drinker lives in the middle house.
The man who smokes Old Golds also keeps snails.  The Ukrainian drinks
tea.  The Norwegian resides in the first house on the left.  The
Chesterfields smoker lives next door to the fox owner.  The Lucky
Strike smoker drinks orange juice.  The Japanese smokes Parliaments.
The horse owner lives next to the Kools smoker, whose house is yellow.
The Norwegian lives next to the blue house.

Now, who drinks water?  And who owns the zebra?

HOUSE | Nationality | Color  | Pet    | Drink        | Smokes
--------------------------------------------------------------------
  1   | norwegian   | yellow | fox    | water        | kools
  2   | ukrainian   | blue   | horse  | tea          | chesterfields
  3   | englishman  | red    | snails | milk         | old-golds
  4   | spaniard    | ivory  | dog    | orange-juice | lucky-strikes
  5   | japanese    | green  | zebra  | coffee       | parliaments

>>> clips.PrintFacts()
f-0     (initial-fact)
f-26    (avh (a smokes) (v parliaments) (h 1))
f-27    (avh (a smokes) (v parliaments) (h 2))
f-28    (avh (a smokes) (v parliaments) (h 3))
\end{verbatim}

[... a long list of facts ...]

\begin{verbatim}
f-150   (avh (a color) (v red) (h 5))
For a total of 126 facts.
>>> li = clips.FactList()
>>> for x in li:
...	if str(x) == 'f-52':
...		f52 = x
>>> f52
<Fact 'f-52': fact object at 0x00E6AA10>
>>> print f52.PPForm()
f-52    (avh (a drink) (v tea) (h 2))
\end{verbatim}

You can just copy the program above to a file, say \file{zebra.clp} as
in the example, and follow the same steps to experiment with \pyclips{}
objects and with the CLIPS subsystem.
		% Introduction
% $Id: contents.tex 334 2008-01-12 04:10:30Z Franz $
\chapter{Module Contents\label{pyclips-modulecontents}}

This chapter gives a detailed description of top-level functions and
constants in the \pyclips{} module. It's not intended to be a CLIPS
reference: the official CLIPS Reference Guide still remains the main
source of documentation for this. This Guide will often be referred to
for information about the engine itself, and the user is expected to
know the CLIPS language and structure sufficiently with respect to his
goals.

Although the \pyclips{} user is not supposed to know CLIPS API, it is
advisable to have its reference at least at hand to sometimes understand
\emph{how} \pyclips{} interacts with CLIPS itself. Besides the programmatic
approach offered by \pyclips{} is vastly different to the C API -- with
the occasional exception of the intrinsic logic.

We will first describe the top level functions.

\section{Top Level Functions and Constants\label{pyclips-toplevel}}

\subsection{Constants\label{pyclips-tl-constants}}

Several constants are provided in order to configure CLIPS environment
or to instruct some functions to behave in particular ways.


\subsubsection{Scope Save Constants}

Constants to decide what to save in CLIPS dump files (see the
\function{SaveInstances()}, \function{BSaveInstances()} and
\function{SaveFacts()} functions).

\begin{tableii}{l|l}{constant}{Constant}{Description}
	\lineii{LOCAL_SAVE}{save objects having templates defined
            in current \class{Module}}
	\lineii{VISIBLE_SAVE}{save all objects visible to current
            \class{Module}}
\end{tableii}


\subsubsection{Salience Evaluation Constants}

Constants to tell the underlying engine when salience has to be
evaluated (see the \var{EngineConfig.SalienceEvaluation} property).

\begin{tableii}{l|l}{constant}{Constant}{Description}
	\lineii{WHEN_DEFINED}{evaluate salience on rule definition
            (the default)}
	\lineii{WHEN_ACTIVATED}{evaluate salience on rule activation
            (\emph{dynamic salience})}
	\lineii{EVERY_CYCLE}{evaluate salience on every execution
            cycle (\emph{dynamic salience})}
\end{tableii}


\subsubsection{Conflict Resolution Strategy Constants}

Constants to specify the way the underlying engine should resolve
conflicts among rules having the same salience (see the
\var{EngineConfig.Strategy} property).

\begin{tableii}{l|l}{constant}{Constant}{Description}
	\lineii{DEPTH_STRATEGY}{newly activated rule comes first}
	\lineii{BREADTH_STRATEGY}{newly activated rule comes last}
	\lineii{LEX_STRATEGY}{strategy based on \emph{time tags}
            applied to facts}
	\lineii{MEA_STRATEGY}{strategy based on \emph{time tags}
            applied to facts}
	\lineii{COMPLEXITY_STRATEGY}{newly activated rule comes before
            rules of lower \emph{specificity}}
	\lineii{SIMPLICITY_STRATEGY}{newly activated rule comes before
            rules of higher \emph{specificity}}
	\lineii{RANDOM_STRATEGY}{assign place in agenda randomly}
\end{tableii}

In the table above, the term \emph{specificity} refers to the number
of comparisons in the LHS of a rule.


\subsubsection{Class Definition Default Mode Constants}

Constants to specify default mode used for classes definition. Please
refer to \clipsbpg{} for details about the usage of the two modes, as
the meaning is quite complex and outside the scope of this manual
(see the \var{EngineConfig.ClassDefaultsMode} property).

\begin{tableii}{l|l}{constant}{Constant}{Description}
	\lineii{CONVENIENCE_MODE}{set the default class mode to
            \keyword{convenience}}
	\lineii{CONSERVATION_MODE}{set the default class mode to
            \keyword{conservation}}
\end{tableii}


\subsubsection{Message Handler Type Constants}

Constants to define the execution time, the purpose and the behaviour of
\emph{message handlers}: see function \function{BuildMessageHandler()}
and the following members of the \class{Class} class:
\function{AddMessageHandler()} and \function{FindMessageHandler()}.
The following table summarizes the analogous one found in \clipsbpg{}.

\begin{tableii}{l|l}{constant}{Constant}{Description}
	\lineii{AROUND}{only to set up an environment for the message}
	\lineii{AFTER}{to perform auxiliary work after the primary message}
	\lineii{BEFORE}{to perform auxiliary work before the primary message}
	\lineii{PRIMARY}{to perform most of the work for the message}
\end{tableii}


\subsubsection{Template Slot Default Type Constants}

It's possible to inspect whether or not a \class{Template} slot has been
defined to have a default value, and in case it is, if its default value
is \emph{static} (that is, constant) or \emph{dynamically generated} (for
instance, using a function like \code{gensym}). See the documentation of
\class{Template} for more details.

\begin{tableii}{l|l}{constant}{Constant}{Description}
	\lineii{NO_DEFAULT}{the slot has no default value}
	\lineii{STATIC_DEFAULT}{the default value is a constant}
	\lineii{DYNAMIC_DEFAULT}{the default value is dynamically generated}
\end{tableii}

Notice that \constant{NO_DEFAULT} evaluates to \constant{False}, so it's
legal to use the \function{Template.Slot.HasDefault()} function just to
test the presence of a default value. Please also note that
\constant{NO_DEFAULT} is only returned when the default value for a slot
is set to \code{?NONE} as stated in the \clipsapg{}.


\subsection{Functions\label{pyclips-tl-functions}}

\begin{funcdesc}{AgendaChanged}{}
test whether or not \code{Agenda} has changed since last call.
\end{funcdesc}

\begin{funcdesc}{Assert}{o}
Assert a \class{Fact} (already created or from a string). This perhaps
needs some explanation: CLIPS allows the creation of \code{fact}s based
on \class{Template}s, and in \pyclips{} this is done by instancing a
\code{Fact} with a \class{Template} argument. The resulting \class{Fact}
slots can then be modified and the object can be used to make an
assertion, either by using the \class{Fact} \function{Assert()}
function or this version od \function{Assert()}.
\end{funcdesc}

\begin{funcdesc}{BLoad}{filename}
Load the constructs from a binary file named \var{filename}. Binary
files are not human-readable and contain all the construct information.
\end{funcdesc}

\begin{funcdesc}{BLoadInstances}{filename}
Load \class{Instance}s from binary file named \var{filename}. Binary
files are not human-readable and contain all the construct information.
\end{funcdesc}

\begin{funcdesc}{BSave}{filename}
Save constructs to a binary file named \var{filename}.
\end{funcdesc}

\begin{funcdesc}{BSaveInstances}{filename \optional{, mode=\constant{LOCAL_SAVE}}}
Save \class{Instance}s to binary file named \var{filename}. The
\var{mode} parameter can be one of \constant{LOCAL_SAVE} (for all
\class{Instance}s whose \class{Definstance}s are defined in current
\class{Module}) or \constant{VISIBLE_SAVE} (for all \class{Instance}s
visible to current \class{Module}).
\end{funcdesc}

\begin{funcdesc}{BatchStar}{filename}
Execute commands stored in text file named as specified in \var{filename}.
\end{funcdesc}

\begin{funcdesc}{BrowseClasses}{name}
Print the list of \class{Class}es that inherit from specified one.
\end{funcdesc}

\begin{funcdesc}{Build}{construct}
Build construct given in argument as a string. The string must enclose
a full construct in the CLIPS language.
\end{funcdesc}

\begin{funcdesc}{BuildClass}{name, text \optional{, comment}}
Build a \class{Class} with specified name and body. \var{comment} is
the optional comment to give to the object. This function is the only
one that can be used to create \class{Class}es with multiple
inheritance.
\end{funcdesc}

\begin{funcdesc}{BuildDeffacts}{name, text \optional{, comment}}
Build a \class{Deffacts} object with specified name and body.
\var{comment} is the optional comment to give to the object.
\end{funcdesc}

\begin{funcdesc}{BuildDefinstances}{name, text \optional{, comment}}
Build a \class{Definstances} having specified name and body.
\var{comment} is the optional comment to give to the object.
\end{funcdesc}

\begin{funcdesc}{BuildFunction}{name, args, text \optional{, comment}}
Build a \class{Function} with specified name, arguments and body.
\var{comment} is the optional comment to give to the object. \var{args}
can be either a blank-separated string containing argument names, or
a sequence of strings corresponding to argument names. Such argument
names should be coherent to the ones used in the function body (that is,
\var{text}). The argument list, if expressed as a string, should
\emph{not} be surrounded by brackets. \var{None} can also be used as
the argument list if the function has no arguments.
\end{funcdesc}

\begin{funcdesc}{BuildGeneric}{name, text \optional{, comment}}
Build a \class{Generic} with specified name and body. \var{comment} is
the optional comment to give to the object.
\end{funcdesc}

\begin{funcdesc}{BuildGlobal}{name, \optional{, value}}
Build a \class{Global} variable with specified \var{name} and \var{value}.
The \var{value} parameter can be of any of the types supported by CLIPS:
it can be expressed as a Python value (with type defined in Python: the
module will try to pass to CLIPS a value of an according type), but for
types that normally do not exist in Python (such as \class{Symbol}s) an
explicit conversion is necessary. If the \var{value} is omitted, then
the module assigns \var{Nil} to the variable.
\end{funcdesc}

\begin{funcdesc}{BuildInstance}{name, defclass \optional{, overrides}}
Build an \class{Instance} of given \class{Class} overriding specified
\code{slots}. If no \code{slot} is specified to be overridden, then the
\class{Instance} will assume default values.
\end{funcdesc}

\begin{methoddesc}{BuildMessageHandler}{name, class, args, text \optional{, type, comment}}
Add a new \emph{message handler} to the supplied class, with specified
name, body (the \var{text} argument) and argument list: this can be
specified either as a sequence of variable names or as a single string
of whitespace separated variable names. Variable names (expressed as
strings) can also be \emph{wildcard parameters}, as specified in the
\clipsbpg{}. The \var{type} parameter should be one of \var{AROUND},
\var{AFTER}, \var{BEFORE}, \var{PRIMARY} defined at the module level:
if omitted it will be considered as \var{PRIMARY}. The body must be
enclosed in brackets, as it is in CLIPS syntax. The function returns
the \emph{index} of the \emph{message handler} within the specified
\class{Class}.
\end{methoddesc}

\begin{funcdesc}{BuildModule}{name \optional{, text, comment}}
Build a \class{Module} with specified name and body. \var{comment} is
the optional comment to give to the object. The current \class{Module}
is set to the new one.
\end{funcdesc}

\begin{funcdesc}{BuildRule}{name, lhs, rhs \optional{, comment}}
Build a \class{Rule} object with specified name and body. \var{comment}
is the optional comment to give to the object. The \var{lhs} and
\var{rhs} parameters correspond to the \emph{left-hand side} and
\emph{right-hand side} of a \class{Rule}.
\end{funcdesc}

\begin{funcdesc}{BuildTemplate}{name, text \optional{, comment}}
Build a \class{Template} object with specified name and body.
\var{comment} is the optional comment to give to the object.
\end{funcdesc}

\begin{funcdesc}{Call}{func, args}
Call a CLIPS internal \class{Function} with the given argument string.
The \var{args} parameter, in its easiest form, can be a list of arguments
separated by blank characters using CLIPS syntax. There are other forms
that can be used, depending on how many arguments the called function
requires: if it accepts a single argument, the caller can just specify
the argument\footnote{It must be of a type compatible with CLIPS. If a
string is supplied, however, it will be considered as a list of arguments
separated by whitespace: in order to explicitly pass a string it has
either to be converted or to be specified surrounded by double quotes.}
possibly cast using one of the \emph{wrapper classes} described below. When
the function accepts multiple arguments it is possible to specify them
as a sequence of values (either a list or a tuple) of basic\footnote{Complex
values, as \emph{multifield}, are not supported: CLIPS does not allow
external calls with non-constant arguments and there is no possibility
to build a \emph{multifield} in place without an explicit function call.}
values. It is always preferable to convert these values using the
\emph{wrapper classes} in order to avoid ambiguity, especially in case of
string arguments.
\end{funcdesc}

\begin{funcdesc}{ClassList}{}
Return the list of \class{Class} names.
\end{funcdesc}

\begin{funcdesc}{Clear}{}
Clear current \class{Environment}.
\end{funcdesc}

\begin{funcdesc}{ClearPythonFunctions}{}
Unregister all user defined Python functions from \pyclips{}.
\end{funcdesc}

\begin{funcdesc}{ClearFocusStack}{}
Clear focus stack.
\end{funcdesc}

\begin{funcdesc}{CurrentEnvironment}{}
Return an \class{Environment} object representing current CLIPS
\emph{environment}. This is useful for switching between \emph{environment}s
in a \pyclips{} session. Please note that almost all \emph{environment}
operations are disallowed on the returned object until another
\class{Environment} is selected as current: all operations on current
\class{Environment} should be performed using the \emph{top level}
module functions.
\end{funcdesc}

\begin{funcdesc}{CurrentModule}{}
Return current module as a \class{Module} object.
\end{funcdesc}

\begin{funcdesc}{DeffactsList}{}
Return a list of \class{Deffacts} names in current \class{Module}.
\end{funcdesc}

\begin{funcdesc}{DefinstancesList}{}
Retrieve list of all \class{Definstances} names.
\end{funcdesc}

\begin{funcdesc}{Eval}{expr}
Evaluate expression passed as argument. Expressions that only have
\emph{side effects} (e.g. \code{printout} expressions) return
\constant{None}.
\end{funcdesc}

\begin{funcdesc}{ExternalTracebackEnabled}{}
Return \constant{True} if functions called from within CLIPS using the
engine function \code{python-call} will print a standard traceback to
\var{sys.stderr} on exceptions, \constant{False} otherwise. This function
is retained for backwards compatibility only: please use the
\var{ExternalTraceback} flag of the \var{DebugConfig} object to enable or
disable this feature instead.
\end{funcdesc}

\begin{funcdesc}{FactList}{}
Return list of \class{Fact}s in current \class{Module}.
\end{funcdesc}

\begin{funcdesc}{FactListChanged}{}
Test whether \class{Fact} list is changed since last call.
\end{funcdesc}

\begin{funcdesc}{FindClass}{name}
Find a \class{Class} by name.
\end{funcdesc}

\begin{funcdesc}{FindDeffacts}{name}
Find a \class{Deffacts} by name.
\end{funcdesc}

\begin{funcdesc}{FindDefinstances}{name}
Find a \class{Definstances} by name.
\end{funcdesc}

\begin{funcdesc}{FindFunction}{name}
Find a \class{Function} by name.
\end{funcdesc}

\begin{funcdesc}{FindGeneric}{name}
Find a \class{Generic} function by name.
\end{funcdesc}

\begin{funcdesc}{FindGlobal}{name}
Find a \class{Global} variable by name.
\end{funcdesc}

\begin{funcdesc}{FindInstance}{name}
Find an \class{Instance} in all \class{Module}s (including imported).
\end{funcdesc}

\begin{funcdesc}{FindInstanceLocal}{name}
Find an \class{Instance} in non imported \class{Module}s.
\end{funcdesc}

\begin{funcdesc}{FindModule}{name}
Find a \class{Module} in list by name.
\end{funcdesc}

\begin{funcdesc}{FindRule}{name}
Find a \class{Rule} by name.
\end{funcdesc}

\begin{funcdesc}{FindTemplate}{name}
Find a \class{Template} by name.
\end{funcdesc}

\begin{funcdesc}{FocusStack}{}
Get list of \class{Module} names in focus stack.
\end{funcdesc}

\begin{funcdesc}{FunctionList}{}
Return the list of \class{Function} names.
\end{funcdesc}

\begin{funcdesc}{GenericList}{}
Return the list of \class{Generic} names.
\end{funcdesc}

\begin{funcdesc}{GlobalList}{}
Return the list of \class{Global} variable names.
\end{funcdesc}

\begin{funcdesc}{GlobalsChanged}{}
Test whether or not \class{Global} variables have changed since last
call.
\end{funcdesc}

\begin{funcdesc}{InitialActivation}{}
Return first \class{\class{Activation}} object in current CLIPS
\class{Environment}.
\end{funcdesc}

\begin{funcdesc}{InitialClass}{}
Return first \class{Class} in current CLIPS \class{Environment}.
\end{funcdesc}

\begin{funcdesc}{InitialDeffacts}{}
Return first \class{Deffacts} in current CLIPS \class{Environment}.
\end{funcdesc}

\begin{funcdesc}{InitialDefinstances}{}
Return first \class{Definstances} in current CLIPS \class{Environment}.
\end{funcdesc}

\begin{funcdesc}{InitialFact}{}
Return first \class{Fact} in current CLIPS \class{Environment}.
\end{funcdesc}

\begin{funcdesc}{InitialFunction}{}
Return first \class{Function} in current CLIPS \class{Environment}.
\end{funcdesc}

\begin{funcdesc}{InitialGeneric}{}
Return first \class{Generic} in current CLIPS \class{Environment}.
\end{funcdesc}

\begin{funcdesc}{InitialGlobal}{}
Return first \class{Global} variable in current CLIPS
\class{Environment}.
\end{funcdesc}

\begin{funcdesc}{InitialInstance}{}
Return first \class{Instance} in current CLIPS \class{Environment}.
\end{funcdesc}

\begin{funcdesc}{InitialModule}{}
Return first \class{Module} in current CLIPS \class{Environment}.
\end{funcdesc}

\begin{funcdesc}{InitialRule}{}
Return first \class{Rule} in current CLIPS \class{Environment}.
\end{funcdesc}

\begin{funcdesc}{InitialTemplate}{}
Return first \class{Template} in current CLIPS \class{Environment}.
\end{funcdesc}

\begin{funcdesc}{InstancesChanged}{}
Test if \class{Instance}s have changed since last call.
\end{funcdesc}

\begin{funcdesc}{Load}{filename}
Load constructs from the specified file named \var{filename}.
\end{funcdesc}

\begin{funcdesc}{LoadFacts}{filename}
Load \class{Fact}s from the specified file named \var{filename}.
\end{funcdesc}

\begin{funcdesc}{LoadFactsFromString}{s}
Load \class{Fact}s from the specified string.
\end{funcdesc}

\begin{funcdesc}{LoadInstances}{filename}
Load \class{Instance}s from file named \var{filename}.
\end{funcdesc}

\begin{funcdesc}{LoadInstancesFromString}{s}
Load \class{Instance}s from the specified string.
\end{funcdesc}

\begin{funcdesc}{MessageHandlerList}{}
Return list of \class{MessageHandler} constructs.
\end{funcdesc}

\begin{funcdesc}{MethodList}{}
Return the list of all methods.
\end{funcdesc}

\begin{funcdesc}{ModuleList}{}
Return the list of \class{Module} names.
\end{funcdesc}

\begin{funcdesc}{PopFocus}{}
Pop focus.
\end{funcdesc}

\begin{funcdesc}{PrintAgenda}{}
Print \class{Rule}s in \code{Agenda} to standard output.
\end{funcdesc}

\begin{funcdesc}{PrintBreakpoints}{}
Print a list of all breakpoints to standard output.
\end{funcdesc}

\begin{funcdesc}{PrintClasses}{}
Print a list of all \class{Class}es to standard output.
\end{funcdesc}

\begin{funcdesc}{PrintDeffacts}{}
Print a list of all \class{Deffacts} to standard output.
\end{funcdesc}

\begin{funcdesc}{PrintDefinstances}{}
Print a list of all \class{Definstances} to standard output.
\end{funcdesc}

\begin{funcdesc}{PrintFacts}{}
Print \class{Fact}s to standard output.
\end{funcdesc}

\begin{funcdesc}{PrintFocusStack}{}
Print focus stack to standard output.
\end{funcdesc}

\begin{funcdesc}{PrintFunctions}{}
Print a list of all \class{Function}s to standard output.
\end{funcdesc}

\begin{funcdesc}{PrintGenerics}{}
Print list of \class{Generic} functions to standard output.
\end{funcdesc}

\begin{funcdesc}{PrintGlobals}{}
print a list of \class{Global} variables to standard output
\end{funcdesc}

\begin{funcdesc}{PrintInstances}{\optional{class}}
Print a list of \class{Instance}s to standard output. If the \var{class}
argument is omitted, all \class{Instance}s in the subsystem will be
shown. The \var{class} parameter can be a \class{Class} object or a
string containing a \class{Class} name.
\end{funcdesc}

\begin{funcdesc}{PrintMessageHandlers}{}
Print a list of all \class{MessageHandler}s.
\end{funcdesc}

\begin{funcdesc}{PrintModules}{}
Print a list of \class{Module}s to standard output.
\end{funcdesc}

\begin{funcdesc}{PrintRules}{}
Print a list of \class{Rule}s to standard output.
\end{funcdesc}

\begin{funcdesc}{PrintSubclassInstances}{\optional{class}}
Print subclass \class{Instance}s to standard output for the
\class{Class} specified. If the \var{class} argument is omitted, all
instances in the subsystem will be shown. The \var{class} parameter can
be a string containing a \class{Class} name or a \class{Class} object.
\end{funcdesc}

\begin{funcdesc}{PrintTemplates}{}
Print \class{Template} names to standard output.
\end{funcdesc}

\begin{funcdesc}{RefreshAgenda}{}
Refresh \code{Agenda} \class{Rule}s for current \class{Module}.
\end{funcdesc}

\begin{funcdesc}{RegisterPythonFunction}{callable \optional{, name}}
Register the function \function{callable} for use within CLIPS via the
engine function \code{python-call}. If the parameter \var{name} of type
\class{str} is not given, then the \var{__name__} attribute of the
first argument will be used. \var{name} is the name that will be used
in CLIPS to refer to the function. See appendix for a more detailed
explanation.
\end{funcdesc}

\begin{funcdesc}{ReorderAgenda}{}
Reorder \code{Agenda} \class{Rule}s for current \class{Module}.
\end{funcdesc}

\begin{funcdesc}{Reset}{}
Reset current \class{Environment}.
\end{funcdesc}

\begin{funcdesc}{RestoreInstancesFromString}{s}
Restore \class{Instance}s from the specified string.
\end{funcdesc}

\begin{funcdesc}{RuleList}{}
Return a list of \class{Rule} names in current \class{Module}.
\end{funcdesc}

\begin{funcdesc}{Run}{\optional{limit}}
Execute \class{Rule}s up to \var{limit} (which is an \class{int} if
given). If \var{limit} is omitted, then no limitation is assumed and the
program runs countinuously until all rules are executed. The function
returns the number of rules that have been fired\footnote{This means,
for instance, that continuously using this function and checking whether
or not the result is less than the specified limit can give more control
over the running CLIPS subprogram, eventually giving the ability to
actively check for the end of the program.}.
\end{funcdesc}

\begin{funcdesc}{Save}{filename}
Save constructs to the file specified by \var{filename}. The constructs are
saved in text (human readable) form.
\end{funcdesc}

\begin{funcdesc}{SaveFacts}{filename \optional{, mode=\constant{LOCAL_SAVE}}}
Save current \class{Fact}s to file specified by \var{filename}. The
\var{mode} parameter can be one of \constant{LOCAL_SAVE} (for all
\class{Fact}s whose \class{Template}s are defined in current
\class{Module}) or \constant{VISIBLE_SAVE} (for all \class{Fact}s
visible to current \class{Module}).
\end{funcdesc}

\begin{funcdesc}{SaveInstances}{filename \optional{, mode=\constant{LOCAL_SAVE}}}
Save \class{Instance}s to file specified by \var{filename}. The
\var{mode} parameter can be one of \constant{LOCAL_SAVE} (for all
\class{Instance}s whose \class{Definstance}s are defined in current
\class{Module}) or \constant{VISIBLE_SAVE} (for all \class{Instance}s
visible to current \class{Module}).
\end{funcdesc}

\begin{funcdesc}{SendCommand}{cmd \optional{, verbose=\constant{False}}}
Send a command to the underlying CLIPS engine, as if it was typed at the
console in an interactive CLIPS session. This command could actually be
useful when embedding a CLIPS shell in a Python program. Please note that
other input than commands, in such a case, should be entered using the
\var{StdinStream} input stream. If \var{verbose} is set to \constant{True}
the possible\footnote{Except for the CLIPS printing functions, as for
instance \code{printout}, that issue an output even when the flag is not
set.} command output is sent to the appropriate output stream.
\end{funcdesc}

\begin{funcdesc}{SetExternalTraceback}{\optional{enabled=\constant{True}}}
Allow or disallow functions called from within the CLIPS engine using
\code{python-call} to print a standard traceback to \var{sys.stderr} in
case an exception occurs. Please note that this does not mean that a
real exception arises, as there is no possibility to catch Python
exceptions in CLIPS code. In such case all failing Python functions will
return the symbol \code{FALSE} to CLIPS\footnote{This is not always an
error condition because a function can intentionally return boolean
values to CLIPS. However the CLIPS engine will report an error message
which can be read from the error stream.}. This behaviour is initially
set to \constant{False}, as it is useful only for debugging purposes.
This function is retained for backwards compatibility only: please use the
\var{ExternalTraceback} flag of the \var{DebugConfig} object to enable or
disable this feature instead.
\end{funcdesc}

\begin{funcdesc}{ShowGlobals}{}
Print list of \class{Global} variables and their values to standard
output (the \function{PrintGlobals()} functions only prints out
\class{Global} variable names).
\end{funcdesc}

\begin{funcdesc}{TemplateList}{}
Return a list of \class{Template} names.
\end{funcdesc}

\begin{funcdesc}{UnregisterPythonFunction}{name}
Remove the function referred as \var{name} within the CLIPS engine from
the set of functions that can be called via \code{python-call} calls.
\end{funcdesc}


Among other exception types, arising in cases that can also occur in
Python, the \pyclips{} module can raise exceptions specific to CLIPS
identified by the following:

\begin{excdesc}{ClipsError}
Exception raised when an operation fails in the CLIPS subsystem: normally
it occurs when CLIPS finds an error, when an iteration is over or when an
invalid value is passed to CLIPS. This exception is accompanied by
explanatory text preceded by an alphanumeric code that can be used
to programmatically identify the error.
\end{excdesc}

\begin{excdesc}{ClipsMemoryError}
Severe memory error raised when the CLIPS subsystem is unable to allocate
the needed memory. In normal circumstances, when an error of this type
occurs, the CLIPS system has become inconsistent and the only way to
recover is exiting. This exception is raised only in order to allow a
developer to notify the user of the impossibility to continue.
\end{excdesc}


\begin{seealso}
\sclipsapg{}
\sclipsbpg{}
\end{seealso}


	% Module Contents
% $Id: objects.tex 347 2008-02-25 00:53:43Z Franz $
\chapter{Classes and Objects\label{pyclips-objects}}

As previously stated in the introduction, \pyclips{} provides classes and
objects to access CLIPS ``\emph{entities}''. It could be preferable to
refer to counterparts ``living'' in the CLIPS subsystem as \emph{entities}
than as \emph{objects}, because it is common practice in OOP to give
the name of ``\emph{objects}'' to class instances: since CLIPS has its
own object oriented structure (in fact there are \emph{classes} in
CLIPS, and therefore also \emph{instances} of these), calling these
structures simply \emph{objects} may generate confusion.

Entities in CLIPS are generated by \emph{constructs}, one for each type
of entity. In Python, the common way to create something is to instance
an \code{object} for a certain \keyword{class}. So it seemed straightforward
to make a class for each of these entities, and to substitute the constructs
used in CLIPS to create entities with \emph{factory functions}. These
functions are defined at module level, and have names of the type
\function{Build\emph{Entity}()} where \function{\emph{Entity}} is the
type of entity that has to be created. The only exception for this are
\class{Fact} objects, which are created in several ways from
\class{Template}s or \function{Assert}ions.

There is another way to create entities in the CLIPS subsystem, that is
directly using the \function{Build()} function with a full CLIPS
construct as string argument. However, this function does not return
anything to the caller, so the created entity has to be sought after
creation to obtain a reference.

The \function{Build\emph{Entity}()} functions and the \function{Assert()}
function return objects of proper types (whose detailed list is given
below) which shadow the corresponding entities in the CLIPS space.

\note{Many objects in \pyclips{} have common features\footnote{The
current \pyclips{} implementation still does not make use of inheritance,
although it's likely that a future release will do.}, such as a
\emph{factory function} as stated above, or methods returning their name
or their so-called \emph{pretty-print form}: in the following detailed
documentation only the first occurrence of a feature will be described
thoroughly.}



\section{Wrapper Classes\label{pyclips-cl-wrapper}}

There are some simple classes that deserve a special mention in the
\pyclips{} module, used to represent in Python namespace the basic types
in CLIPS. These \emph{wrappers} are used to differentiate values that
CLIPS returns from other values that live in the Python space.
However these classes are equivalent to their Python counterparts, and
there is no need to pass objects converted to these classes to the module
functions. Here is a list containing the class names and their
equivalents in Python:

\begin{tableiii}{l|l|l}{class}{Class}{Type}{Python Equivalent}
	\lineiii{Integer}{\var{ClipsIntegerType}}{\class{int}}
	\lineiii{Float}{\var{ClipsFloatType}}{\class{float}}
	\lineiii{String}{\var{ClipsStringType}}{\class{str}}
	\lineiii{Symbol}{\var{ClipsSymbolType}}{\class{str}}
	\lineiii{InstanceName}{\var{ClipsInstanceNameType}}{\class{str}}
	\lineiii{Multifield}{\var{ClipsMultifieldType}}{\class{list}}
\end{tableiii}

A special object named \var{Nil} is defined, and is equivalent to
\code{Symbol('nil')} in comparisons and slot assignments. It is provided
to make code more readable in such situations. It has to be noticed that
also \var{Nil} evaluates to \var{False} in boolean tests: this also
yields for the explicit \code{Symbol('nil')} and \code{Symbol('FALSE')}
definitions\footnote{In this \class{Symbol} is different from \class{str}
as only the empty string evaluates to false in Python. However, it seemed
closer to the assumption that symbols in CLIPS are not to be considered
as ``\emph{literals}'' (they are more similar to implicitly defined
variables) to implement such behaviour, that can be reverted with an
explicit conversion to \class{str}.}.


\section{Template\label{pyclips-cl-Template}}

\class{Template}s are used to build \class{Fact} objects, that is, they
provide a systematic way to construct \class{Fact}s sharing a common
pattern, and the only way to define \class{Fact}s that have named
\class{Slots} (the equivalent of record \emph{fields} or \emph{structure
members} in other programming languages).

\begin{classdesc*}{Template}

This represents a copy of a \code{deftemplate} construct in the CLIPS
subsystem, and not a true \code{deftemplate} entity. More than one
\class{Template} object in Python can refer to the same
\code{deftemplate} entity in the CLIPS subsystem.

\begin{methoddesc}{BuildFact}{}
Build a \class{Fact} object using this \class{Template} without
asserting it. The created \class{Fact} \var{Slots} can be modified and
the \class{Fact} asserted using its \function{Assert} method.
\end{methoddesc}

\begin{memberdesc}[property]{Deletable}
Read-only property to verify if this \class{Template} can be deleted
from the CLIPS subsystem.
\end{memberdesc}

\begin{methoddesc}{InitialFact}{}
Return initial \class{Fact} in list created using this \class{Template}.
\end{methoddesc}

\begin{memberdesc}[property]{Module}
Read-only property to retrieve the CLIPS name of the \class{Module}
where the \class{Template} is defined.
\end{memberdesc}

\begin{memberdesc}[property]{Name}
Read-only property returning the name in CLIPS of this \class{Template}.
The name identifies this entity in the CLIPS subsystem, and has nothing
to do with the name given to the corresponding object in Python.
\end{memberdesc}

\begin{methoddesc}{Next}{}
Return next\footnote{CLIPS stores its objects (or entities) in ordered
lists, so it makes sense to ``iterate'' over these lists. However this
implementation of \pyclips{} does not implement \emph{iterators} (as
known in Python) on these classes: a way to do this is currently under
examination.} \class{Template} in the list of all \class{Template}s.
\var{None} is returned at the end of the list.
\end{methoddesc}

\begin{methoddesc}{NextFact}{fact}
Return next \class{Fact} in list created using this \class{Template},
using the supplied \var{fact} as offset.
\end{methoddesc}

\begin{methoddesc}{PPForm}{}
Return the \emph{pretty-print form} of this \class{Template}.
\emph{Pretty-print forms} are often the code necessary to build a
construct in CLIPS, formatted in a way that makes it quite readable.
The result of the \function{PPForm()} method can be used as the argument
for the \function{Build()} top level function to rebuild the construct
once the \class{Environment} has been cleared\footnote{Actually
\emph{pretty-print forms} use fixed size buffers to build the representing
string: when such a form is too complex, the default buffer size of 8192
bytes can be insufficient. In this case the \emph{PPBufferSize} property
of the \emph{Memory} object can be used to allow the creation of properly
sized buffers.}.
\end{methoddesc}

\begin{methoddesc}{Remove}{}
Remove the entity corresponding to this \class{Template} from the CLIPS
subsystem. This does not remove the corresponding Python object that has
instead to be deleted via the \keyword{del} statement or garbage
collected.
\end{methoddesc}

\begin{memberdesc}[property]{Slots}
\class{Deftemplate} \code{slots} information. This is itself an
object, having many methods, and deserves a special explaination.
\begin{methoddesc}{AllowedValues}{name}
Return a list of allowed values for \code{slot} specified by \var{name}.
\end{methoddesc}
\begin{methoddesc}{Cardinality}{name}
Return \emph{cardinality} for \code{slot} specified by \var{name}.
\end{methoddesc}
\begin{methoddesc}{DefaultValue}{name}
Return \emph{cardinality} for \code{slot} specified by \var{name}.
\end{methoddesc}
\begin{methoddesc}{Exists}{name}
Return \constant{True} if \code{slot} specified by \var{name} exists,
\constant{False} otherwise.
\end{methoddesc}
\begin{methoddesc}{HasDefault}{name}
Return one of the following values: \constant{NO_DEFAULT} if the
default value is set to \code{?NONE}, \constant{STATIC_DEFAULT} when
the default value is static and \constant{DYNAMIC_DEFAULT} when it
is dynamically generated (eg. \code{gensym}).
\end{methoddesc}
\begin{methoddesc}{IsMultifileld}{name}
Return \constant{True} if \code{slot} specified by \var{name} is
a \code{Multifield} value, \constant{False} otherwise.
\end{methoddesc}
\begin{methoddesc}{IsSinglefield}{name}
Return \constant{True} if \code{slot} specified by \var{name} is
a single field value, \constant{False} otherwise.
\end{methoddesc}
\begin{methoddesc}{Names}{}
Return the list of \code{slot} names.
\end{methoddesc}
\begin{methoddesc}{Range}{name}
Return \emph{numerical range information} for \code{slot} specified by
\var{name}.
\end{methoddesc}
\begin{methoddesc}{Types}{name}
Return names of \emph{primitive types} for \code{slot} specified by
\var{name}.
\end{methoddesc}
\end{memberdesc}

\begin{memberdesc}[property]{Watch}
Read-only property to verify if this \class{Template} is being watched.
\end{memberdesc}

The name of this entity in CLIPS is also returned by the string coercion
function. The \emph{factory function} for \class{Template}s is
\function{BuildTemplate()}, which has been discussed above.

\end{classdesc*}



\section{Fact\label{pyclips-cl-Fact}}

\class{Fact}s are one of the main entities in CLIPS, since it is
whether a \class{Fact} exists or not of that drives the subsystem in
the decision to fire or not certain \class{Rule}s. \class{Fact}s, as
seen above, can be created in several ways, that is either by directly
asserting sentences in string form, or by building them first from
\class{Template}s and then asserting them.


\begin{classdesc*}{Fact}

This represents a copy of a fact definition in the CLIPS subsystem, and
not a true fact entity. More than one Fact objects in Python can refer to
the same fact entity in the CLIPS subsystem. Many CLIPS functions return
a \class{Fact} object, but most \class{Fact} objects obtained from CLIPS
are \emph{read-only}\footnote{Actually, the main rule is that if a
\class{Fact} has been \function{Assert}ed then it is read-only. Note that
all \emph{shadow representations} of CLIPS asserted \code{fact} entities
are read-only.}. Read-only \class{Fact}s cannot be reasserted or modified
in \var{Slots}, and are provided for ``informational'' purposes only.

The argument can be a string with the same format of the
\function{Assert()} function seen in the previous chapter: in this case
the fact is created and asserted. Otherwise the argument can be a
\class{Template} object, and in this case the resulting \class{Fact} can
be modified and then asserted via the \function{Assert()} member
function.

\begin{methoddesc}{Assert}{}
Assert this \class{Fact}. Only \class{Fact}s that have been constructed
from \class{Template}s can be \function{Assert}ed using this method:
read-only \class{Fact}s can only be inspected with the other
methods/properties.
\end{methoddesc}

\begin{methoddesc}{AssignSlotDefaults}{}
Assign default values to \var{Slots} of this \class{Fact}.
\end{methoddesc}

\begin{methoddesc}{CleanPPForm}{}
Return only the second part of this \class{Fact}s \emph{pretty-print form}
-- which can be used to build the \class{Fact} itself as described above.
\end{methoddesc}

\begin{memberdesc}{Exists}{}
Is \constant{True} if this \class{Fact} has been asserted (and never
retracted), \constant{False} otherwise.
\end{memberdesc}

\begin{memberdesc}[property]{ImpliedSlots}
The list of all \emph{implied} \var{Slots} for this \class{Fact}.
\end{memberdesc}

\begin{memberdesc}[property]{Index}
Read-only property returning the index in CLIPS of this \class{Fact}. As
for other entities the \var{Name} is a unique identifier, as is the
\var{Index} for \class{Fact}s.
\end{memberdesc}

\begin{methoddesc}{Next}{}
Return next \class{Fact} in the list of all \class{Fact}s. This list is
not based on \class{Module}s, but global to the CLIPS subsystem.
\end{methoddesc}

\begin{methoddesc}{PPForm}{}
Return the \emph{pretty-print form} of this \class{Fact}. In this case,
only the second part of the returned string (the one between parentheses)
can be used to build the \class{Fact} via the \function{Assert()}
function\footnote{This is also not always true: as said before, there is
no way to \function{Assert} \class{Fact}s that have named slots using a
string if there is not a \code{deftemplate} for this kind of
\class{Fact}. However, once a \class{Template} with the specified slots
has been created, this becomes possible.}.
\end{methoddesc}

\begin{methoddesc}{PPrint}{\optional{ignoredefaults}}
Print the \class{Fact} to the standard output. When \var{ignoredefaults}
is set to \constant{True} (the default), slots containing the default
values are omitted.
\end{methoddesc}

\begin{memberdesc}[property]{Relation}
Return only the name of the \emph{relation} that identifies this
\class{Fact}\footnote{The authors of CLIPS call a \emph{relation} the first
field of the \class{Fact} itself, although it is not needed to actually
represent a real relationship.} as a \class{Symbol}.
\end{memberdesc}

\begin{methoddesc}{Retract}{}
Retract this \class{Fact}: in other words, remove the corresponding
entity from the CLIPS subsystem. As in \function{Remove()} seen above,
this does not delete the corresponding Python object. \var{None} is
returned at the end of the list.
\end{methoddesc}

\begin{memberdesc}[property]{Slots}
Dictionary of \class{Fact} \var{Slots}. This member \emph{behaves} like
a \class{dict}, but is not related to such objects. In fact, the values
of \code{slots} are accessible using a \class{dict}-like syntax (square
brackets), but not all the members of \class{dict} are implemented.
\end{memberdesc}

Please note that \class{Fact}s have slightly different methods than
classes representing other entities in CLIPS: an instance of \class{Fact}
is created using the module-level \function{Assert()} function, and
removed using the \function{Retract()} member function: this syntax,
closer to the original CLIPS form, was seen as the preferred method instead
of using a name such as \function{BuildFact()} for creation and a
\function{Remove()} member because of the particular nature of \class{Fact}
related to other types of entity.

Here is an example of usage of \class{Fact} and \class{Template} objects:

\begin{verbatim}
>>> import clips
>>> clips.Reset()
>>> t0 = clips.BuildTemplate("person", """
    (slot name (type STRING))
    (slot age (type INTEGER))
""", "template for a person")
>>> print t0.PPForm()
(deftemplate MAIN::person "template for a person"
   (slot name (type STRING))
   (slot age (type INTEGER)))

>>> f1 = clips.Fact(t0)
>>> f1_slotkeys = f1.Slots.keys()
>>> print f1_slotkeys
<Multifield [<Symbol 'name'>, <Symbol 'age'>]>
>>> f1.Slots['name'] = "Grace"
>>> f1.Slots['age'] = 24
>>> print f1.PPForm()
f-0     (person (name "Grace") (age 24))
>>> clips.PrintFacts()
f-0     (initial-fact)
>>> f1.Assert()
<Fact 'f-1': fact object at 0x00E0CB10>
>>> print f1.Exists()
True
>>> clips.PrintFacts()
f-0     (initial-fact)
f-1     (person (name "Grace") (age 24))
For a total of 2 facts.
>>> f1.Retract()
>>> print f1.Exists()
False
>>> clips.PrintFacts()
f-0     (initial-fact)
For a total of 1 fact.
\end{verbatim}

Please note that slot names are implemented as \class{Symbol}s, and the
list of \var{Slots} is returned as a \class{Multifield}. Also note that
the \class{Fact} \var{f1}, that has been constructed from a
\class{Template} (and not yet \function{Assert}ed) object and then
modified using the \var{Slots} property, can be \function{Assert}ed while
other \class{Fact}s built from construct strings cannot.

\end{classdesc*}



\section{Deffacts\label{pyclips-cl-Deffacts}}

A \class{Deffacts} is used to modify the ``initial structure'' of a CLIPS
environment, by allowing some \class{Fact}s to be \function{Assert}ed by
default each time the \function{Reset()} function is called.

\begin{classdesc*}{Deffacts}

This represents a copy of a \code{deffacts} construct in the CLIPS
subsystem, and not a true \code{deffacts} entity. More than one
\class{Deffacts} object in Python can refer to the same \code{deffacts}
entity in the CLIPS subsystem.

\begin{memberdesc}[property]{Deletable}
Read-only property to verify if this \class{Deffacts} can be deleted.
\end{memberdesc}

\begin{memberdesc}[property]{Module}
Read-only property to retrieve the CLIPS name of the \class{Module}
where the \class{Deffacts} is defined.
\end{memberdesc}

\begin{memberdesc}[property]{Name}
Read-only property returning the name in CLIPS of this \class{Deffacts}.
\end{memberdesc}

\begin{methoddesc}{Next}{}
Return next \class{Deffacts} in the list of all \class{Deffacts}.
\var{None} is returned at the end of the list.
\end{methoddesc}

\begin{methoddesc}{PPForm}{}
Return the \emph{pretty-print form} of this \class{Deffacts}.
\end{methoddesc}

\begin{methoddesc}{Remove}{}
Remove the entity corresponding to this \class{Deffacts} from the CLIPS
subsystem.
\end{methoddesc}

The name of this entity in CLIPS is also returned by the string coercion
function. The \emph{factory function} for \class{Deffacts} is
\function{BuildDeffacts()}, which has been discussed above.

\end{classdesc*}



\section{Rule\label{pyclips-cl-Rule}}

The construct defines rules to be activated and then \emph{fired} whenever
particular conditions are met. This construct is in fact the counterpart
of the \code{defrule} construct in CLIPS. Normally conditions that fire
\class{Rule}s are \class{Fact}s \function{Assert}ed during a session.

\begin{classdesc*}{Rule}

This represents a copy of a \code{defrule} construct in the CLIPS
subsystem, and not a true \code{defrule} entity. More than one
\class{Rule} object in Python can refer to the same \code{defrule}
entity in the CLIPS subsystem.

\begin{memberdesc}[property]{Breakpoint}
Set or remove a breakpoint from this \class{Rule}.
\end{memberdesc}

\begin{memberdesc}[property]{Deletable}
Read-only property to verify if this \class{Rule} can be deleted.
\end{memberdesc}

\begin{memberdesc}[property]{Module}
Read-only property to retrieve the CLIPS name of the \class{Module}
where the \class{Rule} is defined.
\end{memberdesc}

\begin{memberdesc}[property]{Name}
Read-only property returning the name in CLIPS of this \class{Rule}.
\end{memberdesc}

\begin{methoddesc}{Next}{}
Return next \class{Rule} in the list of all \class{Rule}s. \var{None} is
returned at the end of the list.
\end{methoddesc}

\begin{methoddesc}{PPForm}{}
Return the \emph{pretty-print form} of this \class{Rule}.
\end{methoddesc}

\begin{methoddesc}{PrintMatches}{}
Print partial matches of this \class{Rule} to standard output.
\end{methoddesc}

\begin{methoddesc}{Refresh}{}
Refresh this \class{Rule}.
\end{methoddesc}

\begin{methoddesc}{Remove}{}
Remove the entity corresponding to this \class{Rule} from the CLIPS
subsystem.
\end{methoddesc}

\begin{memberdesc}[property]{WatchActivations}
Set or reset debugging of \emph{activations} for this \class{Rule}.
\end{memberdesc}

\begin{memberdesc}[property]{WatchFirings}
Set or reset debugging of \emph{firings} for this \class{Rule}.
\end{memberdesc}

The name of this entity in CLIPS is also returned by the string coercion
function. The \emph{factory function} for \class{Rule}s is
\function{BuildRule()}, which has been discussed above.

\end{classdesc*}

An example -- derived from the ones present in the standard CLIPS
documentation -- may be useful here:

\begin{verbatim}
>>> clips.Reset()
>>> r1 = clips.BuildRule("duck-rule", "(duck)",
                         "(assert (quack))", "The Duck rule")
>>> print r1.PPForm()
(defrule MAIN::duck-rule "The Duck rule"
   (duck)
   =>
   (assert (quack)))

>>> clips.PrintFacts()
f-0     (initial-fact)
For a total of 1 fact.
>>> clips.PrintRules()
MAIN:
duck-rule
>>> f1 = clips.Assert("(duck)")
>>> clips.PrintAgenda()
MAIN:
   0      duck-rule: f-1
For a total of 1 activation.
>>> clips.PrintFacts()
f-0     (initial-fact)
f-1     (duck)
For a total of 2 facts.
>>> clips.Run()
>>> clips.PrintFacts()
f-0     (initial-fact)
f-1     (duck)
f-2     (quack)
For a total of 3 facts.
\end{verbatim}



\section{Activation\label{pyclips-cl-Activation}}

\class{Rule} \class{Activation}s are only returned by the CLIPS
subsystem, and cannot be created -- thus there is no \emph{factory
function} for these objects. CLIPS provides \class{Activation} objects
to keep the program flow under control.

\begin{classdesc*}{Activation}

This represents a copy of an \code{activation} object in the CLIPS
subsystem, and not a true \code{activation} entity. More than one
\class{Activation} object in Python can refer to the same
\code{activation} entity in the CLIPS subsystem.


\begin{memberdesc}[property]{Name}
Retrieve \class{Activation} name.
\end{memberdesc}

\begin{methoddesc}{Next}{}
Return next \class{Activation} in the list of all \class{Activation}s.
\var{None} is returned at the end of the list.
\end{methoddesc}

\begin{methoddesc}{PPForm}{}
Return the \emph{pretty-print form} of \class{Activation}.
\end{methoddesc}

\begin{memberdesc}[property]{Salience}
Retrieve \class{Activation} \emph{salience}\footnote{\emph{Salience} is
a value that represents the \emph{priority} of a \code{rule} in CLIPS.}.
\end{memberdesc}

\begin{methoddesc}{Remove}{}
Remove this \class{Activation} from CLIPS.
\end{methoddesc}

The name of this entity in CLIPS is also returned by the string coercion
function.

\end{classdesc*}



\section{Global\label{pyclips-cl-Global}}

\class{Global} objects represent \emph{global variables} in CLIPS, which
are normally built using the \code{defglobal} construct. To define a new
\class{Global} variable the \function{BuildGlobal()} function must be
used, which returns a new object.

\begin{classdesc*}{Global}

A \class{Global} object represents a copy of a \code{defglobal} construct
in the CLIPS subsystem, and not a true \code{defglobal} entity. More
than one \class{Global} object in Python can refer to the same
\code{defglobal} entity in the CLIPS subsystem.

\begin{memberdesc}[property]{Deletable}
Verify if this \class{Global} can be deleted.
\end{memberdesc}

\begin{memberdesc}[property]{Module}
Read-only property to retrieve the CLIPS name of the \class{Module}
where the \class{Global} is defined.
\end{memberdesc}

\begin{memberdesc}[property]{Name}
Retrieve \class{Global} name. The returned value is a \class{Symbol}
containing the name of the global variable in the CLIPS subsystem.
\end{memberdesc}

\begin{methoddesc}{Next}{}
Return next \class{Global} in the list of all global variables. \var{None}
is returned at the end of the list.
\end{methoddesc}

\begin{methoddesc}{PPForm}{}
Return the \emph{pretty-print} form of \class{Global}.
\end{methoddesc}

\begin{methoddesc}{Remove}{}
Remove this \class{Global} from CLIPS subsystem.
\end{methoddesc}

\begin{memberdesc}[property]{Value}
Set or retrieve \class{Global} value. The returned value can be of many
types, depending on the type of value contained in the corresponding
CLIPS global variable.
\end{memberdesc}

\begin{methoddesc}{ValueForm}{}
Return a ``\emph{printed}'' form of \class{Global} value. The
\emph{printed} form is the one that would be used in CLIPS to represent
the variable itself.
\end{methoddesc}

\begin{memberdesc}[property]{Watch}
Set or retrieve \class{Global} debug status.
\end{memberdesc}

Some examples follow to show the use of \class{Global} objects:

\begin{verbatim}
>>> g_x = clips.BuildGlobal("x", 15)
\end{verbatim}

This is equivalent to the CLIPS declaration:

\begin{verbatim}
CLIPS> (defglobal ?*x* = 15)
\end{verbatim}

Some of the \class{Global} methods are illustrated here:

\begin{verbatim}
>>> g_x
<Global 'x': defglobal object at 0x00E09960>
>>> print g_x
x
>>> g_x.Value
<Integer 15>
>>> print g_x.Value
15
>>> print g_x.ValueForm()
?*x* = 15
\end{verbatim}

The name of this entity in CLIPS is also returned by the string coercion
function.

\end{classdesc*}



\section{Function\label{pyclips-cl-Function}}

Objects of this type represent newly defined \emph{functions}
(usually via the CLIPS \code{deffunction} construct) in the CLIPS
subsystem. In fact the \function{BuildFunction()} function described
above, which returns a \class{Function} object, corresponds to the
\code{deffunction} construct.

\begin{classdesc*}{Function}

This represents a copy of a \code{deffunction} construct in the CLIPS
subsystem, and not a true \code{deffunction} entity. More than one
\class{Function} object in Python can refer to the same
\code{deffunction} entity in the CLIPS subsystem.

\begin{methoddesc}{Call}{\optional{*args}}
Call this \class{Function} with the given arguments, if any. If one only
argument is passed and it is a \class{str}, then it is considered a
``list of whitespace separated arguments\footnote{See the syntax for
the toplevel function with the same name.}'' and follows the CLIPS
syntax: in order to pass a single string it has to be explicitly cast
to the \class{String} \emph{wrapper class}. Conversion to \emph{wrapper
classes} is however recommended for all passed arguments.
\end{methoddesc}

\begin{memberdesc}[property]{Deletable}
Verify if this \class{Function} can be deleted.
\end{memberdesc}

\begin{memberdesc}[property]{Module}
Read-only property to retrieve the CLIPS name of the \class{Module}
where the \class{Function} is defined.
\end{memberdesc}

\begin{memberdesc}[property]{Name}
Retrieve \class{Function} name.
\end{memberdesc}

\begin{methoddesc}{Next}{}
Return next \class{Function} in the list of all CLIPS functions. \var{None}
is returned at the end of the list.
\end{methoddesc}

\begin{methoddesc}{PPForm}{}
Return the \emph{pretty-print form} of \class{Function}.
\end{methoddesc}

\begin{methoddesc}{Remove}{}
Remove this \class{Function}.
\end{methoddesc}

\begin{memberdesc}[property]{Watch}
Set or retrieve \class{Function} debug status.
\end{memberdesc}

The name of this entity in CLIPS is also returned by the string coercion
function.

\note{Objects of this class are \emph{callable} themselves using the
syntax \code{object([arg1 [, ... [argN]])}, where the arguments follow
the same rules as in the \function{Call} method.}

\end{classdesc*}



\section{Generic\label{pyclips-cl-Generic}}

\class{Generic}s (in CLIPS called \emph{generic functions}) are similar
to \class{Function}s, but they add \emph{generic programming} capabilities
to the CLIPS system. Python programmers will find them similar to Python
functions, since \emph{overloading} is possible within the corresponding
construct.

Each different implementation (for different argument sets) of a
\emph{generic function} is called a \code{Method}, and the \class{Generic}
class provides several ways to inspect the various \code{Method}s.
\code{Method}s are identified by an \emph{index}.

\begin{classdesc*}{Generic}

This represents a copy of a \code{defgeneric} construct in the CLIPS
subsystem, and not a true \code{defgeneric} entity. More than one
\class{Generic} objects in Python can refer to the same
\code{defgeneric} entity in the CLIPS subsystem.

\begin{methoddesc}{AddMethod}{restrictions, actions\optional{, midx\optional{, comment}}}
Add a \code{Method} to this \class{Generic}. The structure of this
function resembles the one of the \function{Build<entity>()} functions:
in fact this method of \class{Generic} actually implements the
\code{defmethod} construct which is present in CLIPS. For proper
documentation of this construct, see the CLIPS reference: the
\var{restrictions} parameter (which represents the \code{Method}
\emph{parameter restrictions}) must be expressed \emph{without}
parentheses; the \var{actions} parameter must be expressed as in the
\code{defmethod} construct, that is with all the necessary parentheses
pairs. \var{midx} is the \code{Method} index when it has to be forced
(optionally). The example below should be explanatory. \var{restrictions}
can also be expressed as a sequence of tuples, in each of which the
first element is the argument name (with its proper prefix) as a string
and the following ones are the actual restrictions, either in string form
or as CLIPS primitive types -- which can be specified using \pyclips{}
\emph{wrapper classes} types, see above.
\end{methoddesc}

\begin{methoddesc}{Call}{\optional{*args}}
Call this \class{Generic} with the given arguments, if any. If one only
argument is passed and it is a \class{str}, then it is considered a
``list of whitespace separated arguments'' and follows the CLIPS
syntax: in order to pass a single string it has to be explicitly cast
to the \class{String} \emph{wrapper class}. Conversion to \emph{wrapper
classes} is however recommended for all passed arguments.
\end{methoddesc}

\begin{memberdesc}[property]{Deletable}
Verify if this \class{Generic} can be deleted.
\end{memberdesc}

\begin{methoddesc}{InitialMethod}{}
Return the index of first \code{Method} in this \class{Generic}.
\end{methoddesc}

\begin{methoddesc}{MethodDeletable}{midx}
Test whether or not specified \code{Method} can be deleted from this
\class{Generic}.
\end{methoddesc}

\begin{methoddesc}{MethodDescription}{midx}
Return the synopsis of specified \code{Method} \emph{restrictions}.
\end{methoddesc}

\begin{methoddesc}{MethodList}{}
Return the list of \code{Method} indices for this \class{Generic}.
\end{methoddesc}

\begin{methoddesc}{MethodPPForm}{midx}
Return the \emph{pretty-print form} of specified \code{Method}.
\end{methoddesc}

\begin{methoddesc}{MethodRestrictions}{midx}
Return the \emph{restrictions} of specified \code{Method} in this
\class{Generic} object: the \var{midx} parameter must be an
\class{Integer} or \class{int} indicating the \code{Method} index.
\end{methoddesc}

\begin{methoddesc}{MethodWatched}{midx}
Test whether or not specified \code{Method} is being watched.
\end{methoddesc}

\begin{memberdesc}[property]{Module}
Read-only property to retrieve the CLIPS name of the \class{Module}
where the \class{Generic} is defined.
\end{memberdesc}

\begin{memberdesc}[property]{Name}
Retrieve \class{Generic} name.
\end{memberdesc}

\begin{methoddesc}{NextMethod}{midx}
Return the index of next \code{Method} in this \class{Generic} given
the start index as an \class{Integer} or \class{int}.
\end{methoddesc}

\begin{methoddesc}{PPForm}{}
Return the \emph{pretty-print form} of \class{Generic}.
\end{methoddesc}

\begin{methoddesc}{PrintMethods}{}
Print out \code{Method} list for this \class{Generic}.
\end{methoddesc}

\begin{methoddesc}{Remove}{}
Remove this \class{Generic}.
\end{methoddesc}

\begin{methoddesc}{RemoveMethod}{midx}
Remove specified \code{Method} from this \class{Generic}.
\end{methoddesc}

\begin{methoddesc}{UnwatchMethod}{midx}
Deactivate watch on specified \code{Method}.
\end{methoddesc}

\begin{memberdesc}[property]{Watch}
Set or retrieve \class{Generic} debug status.
\end{memberdesc}

\begin{methoddesc}{WatchMethod}{midx}
Activate watch on specified \code{Method}.
\end{methoddesc}

The name of this entity in CLIPS is also returned by the string coercion
function. The \emph{factory function} for \class{Generic}s is
\function{BuildGeneric()}, which has been discussed above.

\note{Objects of this class are \emph{callable} themselves using the
syntax \code{object([arg1 [, ... [argN]])}, where the arguments follow
the same rules as in the \function{Call} method.}

An example for this class follows.

\begin{verbatim}
>>> import clips
>>> addf = clips.BuildGeneric("my-addf", "my generic add function")
>>> addf.AddMethod("(?a STRING)(?b STRING)", "(str-cat ?a ?b)")
>>> addf.AddMethod("(?a INTEGER)(?b INTEGER)", "(+ ?a ?b)")
>>> addf.PrintMethods()
my-addf #1  (STRING) (STRING)
my-addf #2  (INTEGER) (INTEGER)
For a total of 2 methods.
>>> print addf.MethodPPForm(1)
(defmethod MAIN::my-addf
   ((?a STRING)
    (?b STRING))
   (str-cat ?a ?b))

>>> print addf.PPForm()
(defgeneric MAIN::my-addf "my generic add function")

>>> print clips.Eval('(my-addf 5 13)')
18
>>> print clips.Eval('(my-addf "hello,"(my-addf " " "world!"))')
hello, world!
>>> print clips.Eval('(my-addf "hello" 13)')
Traceback (most recent call last):
  File "<pyshell#14>", line 1, in ?
    print clips.Eval('(my-addf "hello" 13)')
  File ".../_clips_wrap.py", line 2472, in Eval
    return _cl2py(_c.eval(expr))
ClipsError: C10: unable to evaluate expression
>>> s = clips.ErrorStream.Read()
>>> print s
[GENRCEXE1] No applicable methods for my-addf.
\end{verbatim}

\end{classdesc*}

Please note how the \emph{error stream} (\var{ErrorStream}) can be used
to retrieve a more explanatory text for the error. The \emph{error
stream} can be very useful during interactive debugging \pyclips{}
sessions to fix errors.



\section{Class\label{pyclips-cl-Class}}

\class{Class} objects are \code{class} definition constructs, the most
important feature of the \emph{COOL}\footnote{Acronym for CLIPS
Object-Oriented Language.} sublanguage of CLIPS. As in other OOP
environments, \code{class}es represent in CLIPS new data types (often
resulting from aggregation of simpler data types) which have particular ways
of being handled. Normally, as in Python, these particular ways are called
\emph{methods}\footnote{Note that the term \code{Method} has been used
for function overloading in the definition of \class{Generic}
functions.}, while in CLIPS they are called \emph{message handlers},
since to apply a method to a CLIPS object (in fact, the \class{Instance}
of a \class{Class} in \pyclips{}) a \emph{message} has to be sent to that
object.

\begin{classdesc*}{Class}

This represents a copy of a \code{defclass} construct in the CLIPS
subsystem, and not a true \code{defclass} entity. More than one
\class{Class} object in Python can refer to the same \code{defclass}
entity in the CLIPS subsystem.

\begin{memberdesc}[property]{Abstract}
Verify if this \class{Class} is \emph{abstract} or not.
\end{memberdesc}

\begin{methoddesc}{AddMessageHandler}{name, args, text \optional{, type, comment}}
Add a new \emph{message handler} to this class, with specified name,
body (the \var{text} argument) and argument list: this can be
specified either as a sequence of variable names or as a single string
of whitespace separated variable names. Variable names (expressed as
strings) can also be \emph{wildcard parameters}, as specified in the
\clipsbpg{}. The \var{type} parameter should be one of \var{AROUND},
\var{AFTER}, \var{BEFORE}, \var{PRIMARY} defined at the module level:
if omitted it will be considered as \var{PRIMARY}. The body must be
enclosed in brackets, as it is in CLIPS syntax. The function returns
the \emph{index} of the \emph{message handler} within this \class{Class}.
\end{methoddesc}

\begin{methoddesc}{AllMessageHandlerList}{}
Return the list of \code{MessageHandler} constructs of this \class{Class}
including the ones that have been inherited from the superclass.
\end{methoddesc}

\begin{methoddesc}{BuildInstance}{name, \optional{overrides}}
Build an \class{Instance} of this \class{Class} with the supplied name
and overriding specified \code{slots}. If no \code{slot} is specified
to be overridden, then the \class{Instance} will assume default values.
\end{methoddesc}

\begin{methoddesc}{BuildSubclass}{name, text \optional{, comment}}
Build a subclass of this \class{Class} with specified name and body.
\var{comment} is the optional comment to give to the object.
\end{methoddesc}

\begin{memberdesc}[property]{Deletable}
Verify if this \class{Class} can be deleted.
\end{memberdesc}

\begin{methoddesc}{Description}{}
Return a summary of \class{Class} description.
\end{methoddesc}

\begin{methoddesc}{InitialInstance}{}
Return initial \class{Instance} of this \class{Class}. It raises an
error if the \class{Class} has no subclass \class{Instance}s.
\end{methoddesc}

\begin{methoddesc}{InitialSubclassInstance}{}
Return initial instance of this \class{Class} including its subclasses. It
raises an error if the \class{Class} has no subclass \class{Instance}s.
\end{methoddesc}

\begin{methoddesc}{IsSubclassOf}{o}
Test whether this \class{Class} is a subclass of specified \class{Class}.
\end{methoddesc}

\begin{methoddesc}{IsSuperclassOf}{o}
Test whether this \class{Class} is a superclass of specified
\class{Class}.
\end{methoddesc}

\begin{methoddesc}{MessageHandlerDeletable}{index}
Return true if specified \code{MessageHandler} can be deleted.
\end{methoddesc}

\begin{methoddesc}{MessageHandlerIndex}{name \optional{, htype}}
Find the specified \code{MessageHandler}, given its \var{name} and type
(as the parameter \var{htype}). If type is omitted, it is considered to
be \var{PRIMARY}.
\end{methoddesc}

\begin{methoddesc}{MessageHandlerName}{index}
Return the name of specified \code{MessageHandler}.
\end{methoddesc}

\begin{methoddesc}{MessageHandlerList}{}
Return the list of \code{MessageHandler} constructs for this
\class{Class}.
\end{methoddesc}

\begin{methoddesc}{MessageHandlerPPForm}{index}
Return the \emph{pretty-print form} of \code{MessageHandler}.
\end{methoddesc}

\begin{methoddesc}{MessageHandlerType}{index}
Return the type of the \code{MessageHandler} specified by the provided
\var{index}.
\end{methoddesc}

\begin{methoddesc}{MessageHandlerWatched}{index}
Return watch status of specified \code{MessageHandler}.
\end{methoddesc}

\begin{memberdesc}[property]{Module}
Read-only property to retrieve the CLIPS name of the \class{Module}
where the \class{Class} is defined.
\end{memberdesc}

\begin{memberdesc}[property]{Name}
Retrieve \class{Class} name.
\end{memberdesc}

\begin{methoddesc}{Next}{}
Return next \class{Class} in the list of all CLIPS \code{classes}.
\var{None} is returned at the end of the list.
\end{methoddesc}

\begin{methoddesc}{NextInstance}{instance}
Return next \class{Instance} of this \class{Class}. Returns \var{None} if
there are no \class{Instance}s left.
\end{methoddesc}

\begin{methoddesc}{NextMessageHandlerIndex}{index}
Return index of next \code{MessageHandler} with respect to the specified
one.
\end{methoddesc}

\begin{methoddesc}{NextSubclassInstance}{instance}
Return next instance of this \class{Class}, including subclasses. Returns
\var{None} if there are no \class{Instance}s left.
\end{methoddesc}

\begin{methoddesc}{PPForm}{}
Return the \emph{pretty-print form} of \class{Class}.
\end{methoddesc}

\begin{methoddesc}{PreviewSend}{msgname}
Print list of \code{MessageHandler}s suitable for specified message.
\end{methoddesc}

\begin{methoddesc}{PrintAllMessageHandlers}{}
Print the list of all \code{MessageHandler}s for this \class{Class}
including the ones that have been inherited from the superclass.
\end{methoddesc}

\begin{methoddesc}{PrintMessageHandlers}{}
Print the list of \code{MessageHandler}s for this \class{Class}.
\end{methoddesc}

\begin{methoddesc}{RawInstance}{name}
Create an empty \class{Instance} of this \class{Class} with specified
name.
\end{methoddesc}

\begin{memberdesc}[property]{Reactive}
Verify if this \class{Class} is \emph{reactive} or not.
\end{memberdesc}

\begin{methoddesc}{Remove}{}
Remove this \class{Class}.
\end{methoddesc}

\begin{methoddesc}{RemoveMessageHandler}{index}
Remove \code{MessageHandler} specified by the provided \var{index}.
\end{methoddesc}

\begin{memberdesc}[property]{Slots}
\class{Class} \code{slots} information. This is itself an object, having
many methods, and deserves a special explaination.
\begin{methoddesc}{AllowedClasses}{name}
Return a list of allowed class names for \code{slot} specified by
\var{name}.
\end{methoddesc}
\begin{methoddesc}{AllowedValues}{name}
Return a list of allowed values for \code{slot} specified by \var{name}.
\end{methoddesc}
\begin{methoddesc}{Cardinality}{name}
Return \emph{cardinality} for \code{slot} specified by \var{name}.
\end{methoddesc}
\begin{methoddesc}{DefaultValue}{name}
Return the default value for \code{slot} specified by \var{name}.
\end{methoddesc}
\begin{methoddesc}{Exists}{name}
Return \constant{True} if \code{slot} specified by \var{name} exists,
\constant{False} otherwise.
\end{methoddesc}
\begin{methoddesc}{ExistsDefined}{name}
Return \constant{True} if \code{slot} specified by \var{name} is defined
in this \class{Class}, \constant{False} otherwise.
\end{methoddesc}
\begin{methoddesc}{Facets}{name}
Return \emph{facet names} for \code{slot} specified by \var{name}.
\end{methoddesc}
\begin{methoddesc}{HasDirectAccess}{name}
Return \constant{True} if \code{slot} specified by \var{name} is directly
accessible, \constant{False} otherwise.
\end{methoddesc}
\begin{methoddesc}{IsInitable}{name}
Return \constant{True} if \code{slot} specified by \var{name} is
\emph{initializable}, \constant{False} otherwise.
\end{methoddesc}
\begin{methoddesc}{IsPublic}{name}
Return \constant{True} if \code{slot} specified by \var{name} is
\emph{public}, \constant{False} otherwise.
\end{methoddesc}
\begin{methoddesc}{IsWritable}{name}
Return \constant{True} if \code{slot} specified by \var{name} is
\emph{writable}, \constant{False} otherwise.
\end{methoddesc}
\begin{methoddesc}{Names}{}
Return the list of \code{slot} names.
\end{methoddesc}
\begin{methoddesc}{NamesDefined}{}
Return the list of \code{slot} names explicitly defined in this
\class{Class}.
\end{methoddesc}
\begin{methoddesc}{Range}{name}
Return \emph{numerical range information} for \code{slot} specified by
\var{name}.
\end{methoddesc}
\begin{methoddesc}{Sources}{name}
Return \emph{source class names} for \code{slot} specified by \var{name}.
\end{methoddesc}
\begin{methoddesc}{Types}{name}
Return names of \emph{primitive types} for \code{slot} specified by
\var{name}.
\end{methoddesc}
\end{memberdesc}

\begin{methoddesc}{Subclasses}{}
Return the names of subclasses of this \class{Class}.
\end{methoddesc}

\begin{methoddesc}{Superclasses}{}
Return the names of superclasses of this \class{Class}.
\end{methoddesc}

\begin{methoddesc}{UnwatchMessageHandler}{index}
Turn off debug for specified \code{MessageHandler}.
\end{methoddesc}

\begin{memberdesc}[property]{WatchInstances}
Set or retrieve debug status for this \class{Class} \class{Instance}s.
\end{memberdesc}

\begin{methoddesc}{WatchMessageHandler}{index}
Turn on debug for specified \code{MessageHandler}.
\end{methoddesc}

\begin{memberdesc}[property]{WatchSlots}
Set or retrieve \code{Slot} debug status.
\end{memberdesc}

The name of this entity in CLIPS is also returned by the string coercion
function. The \emph{factory function} for \class{Class}es is
\function{BuildClass()}, which has been discussed above.

\end{classdesc*}



\section{Instance\label{pyclips-cl-Instance}}

\class{Instance} objects represent \emph{class instances} (that is,
\emph{objects} in the OOP paradigm) that live in the CLIPS subsystem.
Messages can be sent to those objects and values can be set and retrieved
for the \emph{slots} defined in the related \emph{class}, where the
meaning of \code{slot} has been described in the section above.

\begin{classdesc*}{Instance}

This represents a copy of an \code{instance} object in the CLIPS
subsystem, and not a true \code{instance} entity. More than one
\class{Instance} object in Python can refer to the same \code{instance}
entity in the CLIPS subsystem.

\begin{memberdesc}[property]{Class}
Retrieve the \class{Class} of this \class{Instance}: this property
actually refers to a \class{Class} object, so all of its methods are
available.
\end{memberdesc}

\begin{methoddesc}{DirectRemove}{}
Directly remove this \class{Instance}, without sending a message.
\end{methoddesc}

\begin{methoddesc}{GetSlot}{slotname}
Retrieve the value of \code{Slot} specified as argument. The synonym
\function{SlotValue} is retained for readability and compatibility.
Please notice that these functions are provided in order to be more
coherent with the behaviour of CLIPS API, as CLIPS C interface users
know that a function like \function{GetSlot} usually bypasses message
passing, thus accessing \code{slots} directly. The possibilities
offered by \function{GetSlot} are also accessible using the \var{Slots}
property described below.
\end{methoddesc}

\begin{methoddesc}{IsValid}{}
Determine if this \class{Instance} is still valid.
\end{methoddesc}

\begin{memberdesc}[property]{Name}
Retrieve the \class{Instance} name.
\end{memberdesc}

\begin{methoddesc}{Next}{}
Return next \class{Instance} in the list of all CLIPS \code{instances}.
It returns \var{None} if there are no \class{Instance}s left.
\end{methoddesc}

\begin{methoddesc}{PPForm}{}
Return the \emph{pretty-print form} of \class{Instance}.
\end{methoddesc}

\begin{methoddesc}{PutSlot}{slotname, value}
Set the value of specified \code{slot}. The \var{value} parameter should
contain a value of the correct type, if necessary cast to one of the
\emph{wrapper classes} described above if the type could be ambiguous.
The synonym \function{SetSlotValue} is provided for readability and
compatibility. What has been said about \function{GetSlot} also yields
for the hereby described function, as the possibilities offered by
\function{PutSlot} are also accessible using the \var{Slots} property
described below.
\end{methoddesc}

\begin{methoddesc}{Remove}{}
Remove this \class{Instance} (passing a message).
\end{methoddesc}

\begin{methoddesc}{Send}{msg \optional{, args}}
Send the provided \emph{message} with the given arguments to
\class{Instance}. The \var{args} parameter (that is, \emph{message
arguments}), should be a string containing a list of arguments
separated by whitespace, a tuple containing the desired arguments or a
value of a basic type. Also in the second case the tuple elements have
to be of basic types. The function returns a value depending on the
passed message.
\end{methoddesc}

\begin{memberdesc}[property]{Slots}
Dictionary of \class{Instance} \code{slots}. This member \emph{behaves}
like a \class{dict}, but is not related to such objects. In fact, the
values of \code{slots} are accessible using a \class{dict}-like syntax
(square brackets), but not all the members of \class{dict} are
implemented. The functionality of \function{PutSlot} and
\function{GetSlot} is superseded by this property.
\end{memberdesc}

The name of this entity in CLIPS is also returned by the string coercion
function. The \emph{factory function} for \class{Instance}s is
\function{BuildInstance()}, which has been discussed above.

Here is an example of usage of \class{Instance} and \class{Class} objects:

\begin{verbatim}
>>> import clips
>>> clips.Build("""(defclass TEST1
    (is-a USER)
    (slot ts1 (type INSTANCE-NAME))
    (multislot ts2))""")
>>> c = clips.FindClass("TEST1")
>>> print c.PPForm()
(defclass MAIN::TEST1
   (is-a USER)
   (slot ts1
      (type INSTANCE-NAME))
   (multislot ts2))

>>> clips.Reset()
>>> i = clips.BuildInstance("test1", c)
>>> i.Slots['ts2'] = clips.Multifield(['hi', 'there'])
>>> i.Slots['ts1'] = i.Name
>>> print i.PPForm()
[test1] of TEST1 (ts1 [test1]) (ts2 "hi" "there")
\end{verbatim}

\end{classdesc*}



\section{Definstances\label{pyclips-cl-Definstances}}

As there are \code{deffacts} for \code{fact} objects, \code{instances}
are supported in CLIPS by the \code{definstances} construct: it allows
certain default \class{Instance}s to be created each time a
\function{Reset()} is issued. In \pyclips{} this construct is provided
via the \class{Definstances} class.

\begin{classdesc*}{Definstances}

This represents a copy of the \code{definstances} construct in the CLIPS
subsystem, and not a true \code{definstances} entity. More than one
\class{Definstances} object in Python can refer to the same
\code{definstances} entity in the CLIPS subsystem.

\begin{memberdesc}[property]{Deletable}
Verify if this \class{Definstances} can be deleted.
\end{memberdesc}

\begin{memberdesc}[property]{Module}
Read-only property to retrieve the CLIPS name of the \class{Module}
where the \class{Definstances} is defined.
\end{memberdesc}

\begin{memberdesc}[property]{Name}
Retrieve \class{Definstances} name.
\end{memberdesc}

\begin{methoddesc}{Next}{}
Return next \class{Definstances} in the list of all CLIPS
\code{definstances}. \var{None} is returned at the end of the list.
\end{methoddesc}

\begin{methoddesc}{PPForm}{}
Return the \emph{pretty-print form} of this \class{Definstances} object.
\end{methoddesc}

\begin{methoddesc}{Remove}{}
Delete this \class{Definstances} object from CLIPS subsystem.
\end{methoddesc}

The name of this entity in CLIPS is also returned by the string coercion
function. The \emph{factory function} for \class{Definstances} is
\function{BuildDefinstances()}, which has been discussed above.

\end{classdesc*}



\section{Module\label{pyclips-cl-Module}}

\class{Module}s are a way, in CLIPS, to organize constructs, facts and
objects. There is a big difference between \emph{modules} and
\emph{environments}\footnote{Besides the discussion above, also notice
that in a ``pure'' CLIPS session there is no concept of \emph{environment}
at all: the use of environment is reserved to those who embed CLIPS in
another program, such as \pyclips{} users.}: one should think of a
\class{Module} as a \emph{group} of definitions and objects, which can
interoperate with entities that are defined in other \class{Module}s. The
\class{Module} class provides methods, similar to the ones defined at top
level, to directly create entities as part of the \class{Module} itself,
as well as methods to examine \class{Module} contents. Also,
\class{Module} objects have methods that instruct the related CLIPS
\code{module} to become \emph{current}, so that certain operations can
be performed without specifying the \code{module} to which they have to
be applied.

\begin{classdesc*}{Module}

This represents a copy of a \code{defmodule} construct in the CLIPS
subsystem, and not a true \code{defmodule} entity. More than one
\class{Module} object in Python can refer to the same \code{defmodule}
entity in the CLIPS subsystem.

\begin{memberdesc}[property]{Name}
Return the name of this \class{Module}.
\end{memberdesc}

\begin{methoddesc}{Next}{}
Return next \class{Module} in the list of all CLIPS \code{modules}.
\var{None} is returned at the end of the list.
\end{methoddesc}

\begin{methoddesc}{PPForm}{}
Return the \emph{pretty-print form} of this \class{Module}.
\end{methoddesc}

\begin{methoddesc}{SetCurrent}{}
Make the \code{module} that this object refers the current \class{Module}.
\end{methoddesc}

\begin{methoddesc}{SetFocus}{}
Set focus to this \class{Module}.
\end{methoddesc}

For the following methods:
\begin{verbatim}
TemplateList(), FactList(), DeffactsList(), ClassList(), DefinstancesList(),
GenericList(), FunctionList(), GlobalList(), BuildTemplate(),
BuildDeffacts(), BuildClass(), BuildDefinstances(), BuildGeneric(),
BuildFunction(), BuildGlobal(), BuildRule(), BuildInstance(),
PrintTemplates(), PrintDeffacts(), PrintRules(), PrintClasses(),
PrintInstances(), PrintSubclassInstances(), PrintDefinstances(),
PrintGenerics(), PrintFunctions(), PrintGlobals(), ShowGlobals(),
PrintAgenda(), PrintBreakpoints(), ReorderAgenda(), RefreshAgenda()
\end{verbatim}

please refer to the corresponding function defined at module level,
keeping in mind that these methods perform the same task but within the
\class{Module} where they are executed.

The name of this entity in CLIPS is also returned by the string coercion
function. The \emph{factory function} for \class{Module}s is
\function{BuildModule()}, which has been discussed above.

\end{classdesc*}



\section{Environment\label{pyclips-cl-Environment}}

This class represents an \emph{environment}, and implements almost all
the module level functions and classes. The only objects appearing at
\pyclips{} level and \emph{not} at \class{Environment} level are the
CLIPS I/O subsystem \emph{streams}, which are shared with the rest of
the CLIPS engine.

\class{Environment} objects are not a feature of CLIPS sessions (as
stated above), thus there is no way to identify them in CLIPS using
a \emph{symbol}. So \class{Environment} objects do not have a \var{Name}
property. Instead, CLIPS provides a way to identify an
\emph{environment} through an integer called \emph{index}.

\begin{classdesc*}{Environment}

Please refer to top level functions, variables and classes for
information on contents of \class{Environment} objects. The extra
methods and properties follow below.

\begin{memberdesc}[property]{Index}
Retrieve the \emph{index} identifying this \class{Environment}
internally in CLIPS.
\end{memberdesc}

\begin{methoddesc}{SetCurrent}{}
Make the \emph{environment} that this object refers the current
\class{Environment}.
\end{methoddesc}

Further explanations about \class{Environment} objects can be found
in the appendices.

\end{classdesc*}



\section{Status and Configuration Objects\label{pyclips-cl-statusconf}}

As seen in the introduction, there are a couple of objects that can be
accessed to configure the underlying CLIPS engine and to retrieve its
status. These are the \var{EngineConfig} and \var{DebugConfig} objects.
The reason why configuration and status functions have been grouped in
these objects is only cosmetic: in fact there is no counterpart of
\var{EngineConfig} and \var{DebugConfig} in CLIPS. It was seen as convenient
to group configuration and debug functions in two main objects and to
make them accessible mainly as \emph{properties} in Python, instead of
populating the module namespace with too many \emph{get/set} functions.

There is also an object, called \var{Memory}, which gives information
about memory utilization and allow the user to attempt to free memory
used by the CLIPS engine and no longer needed.

A description of what the above objects (which can not be instanced by
the user of \pyclips{}\footnote{Besides removal of class definitions, a
\emph{singleton}-styled implementation mechanism prevents the user from
creating further instances of the objects.}) actually expose follows.


\subsection{Engine Configuration\label{pyclips-cl-statusconf-engine}}

The \var{EngineConfig} object allows the configuration of some features
of the underlying CLIPS engine. Here are the properies provided by
\var{EngineConfig}:

\begin{memberdesc}[property]{AutoFloatDividend}
Reflects the behaviour of CLIPS \code{get/set-auto-float-dividend}. When
\constant{True} the dividend is always considered to be a floating point
number within divisions.
\end{memberdesc}

\begin{memberdesc}[property]{ClassDefaultsMode}
Reflects the behaviour of CLIPS \code{get/set-class-defaults-mode}.
Possible values of this flag are \constant{CONVENIENCE_MODE} and
\constant{CONSERVATION_MODE}. See \clipsapg{} for details.
\end{memberdesc}

\begin{memberdesc}[property]{DynamicConstraintChecking}
Reflects the behaviour of CLIPS \code{get/set-dynamic-constraint-checking}.
When \constant{True}, \emph{function calls} and \emph{constructs} are
checked against constraint violations.
\end{memberdesc}

\begin{memberdesc}[property]{FactDuplication}
Reflects the behaviour of CLIPS \code{get/set-fact-duplication}. When
\constant{True}, \code{facts} can be reasserted when they have already
been asserted\footnote{This does not change the behaviour of the
\class{Fact} class, which prohibits reassertion anyway. However,
\code{facts} that would be asserted through firing of rules and would
generate duplications will not raise an error when this behaviour is
set.}.
\end{memberdesc}

\begin{memberdesc}[property]{IncrementalReset}
Reflects the behaviour of CLIPS \code{get/set-incremental-reset}. When
\constant{True} newly defined \code{rules} are updated according to
current \code{facts}, otherwise new \code{rules} will only be updated by
\code{facts} defined after their construction.
\end{memberdesc}

\begin{memberdesc}[property]{ResetGlobals}
Reflects the behaviour of CLIPS \code{get/set-reset-globals}. When
\constant{True} the \class{Global} variables are reset to their initial
value after a call to \function{Reset()}.
\end{memberdesc}

\begin{memberdesc}[property]{SalienceEvaluation}
Reflects the behaviour of CLIPS \code{get/set-salience-evaluation}. Can
be one of \constant{WHEN_DEFINED}, \constant{WHEN_ACTIVATED},
\constant{EVERY_CYCLE}. See the previous chapter and \clipsapg{} for
more information.
\end{memberdesc}

\begin{memberdesc}[property]{SequenceOperatorRecognition}
Reflects the behaviour of CLIPS
\code{get/set-sequence-operator-recognition}. When \constant{False},
\class{Multifield} values in function calls are treated as a single
argument.
\end{memberdesc}

\begin{memberdesc}[property]{StaticConstraintChecking}
Reflects the behaviour of CLIPS \code{get/set-static-constraint-checking}.
When \constant{True}, \emph{slot values} are checked against constraint
violations.
\end{memberdesc}

\begin{memberdesc}[property]{Strategy}
Reflects \code{get/set-strategy} behaviour. Can be any of the following
values: \constant{RANDOM_STRATEGY}, \constant{LEX_STRATEGY},
\constant{MEA_STRATEGY}, \constant{COMPLEXITY_STRATEGY},
\constant{SIMPLICITY_STRATEGY}, \constant{BREADTH_STRATEGY} or
\constant{DEPTH_STRATEGY}. See the previous chapter and \clipsapg{} for
more information.
\end{memberdesc}


\subsection{Debug Settings\label{pyclips-cl-statusconf-debug}}

The \var{DebugConfig} object provides access to the debugging and trace
features of CLIPS. During a CLIPS interactive session debug and trace
messages are printed on the system console (which maps the \code{wtrace}
I/O \emph{router}). Users of the trace systems will have to poll the
\var{TraceStream} to read the generated messages.

In CLIPS, the process of enabling trace features on some class of
entities is called \emph{to watch} such a class; this naming convention
is reflected in \pyclips{}. Note that specific objects can be
\emph{watched}: many classes have their own \var{Watch} property to
enable or disable debugging on a particular object.

Also, CLIPS provides a facility to log all debug information to physical
files: this is called \emph{to dribble} on a file. \emph{Dribbling} is
possible from \var{DebugConfig} via the appropriate methods.

The names of methods and properties provided by this object are quite
similar to the corresponding commands in CLIPS, so more information
about debugging features can be found in \clipsbpg{}.

\begin{memberdesc}[property]{ActivationsWatched}
Flag to enable or disable trace of \class{Rule} activations and
deactivations.
\end{memberdesc}

\begin{memberdesc}[property]{CompilationsWatched}
Flag to enable or disable trace of construct definition progress.
\end{memberdesc}

\begin{methoddesc}{DribbleActive}{}
Tell whether or not \emph{dribble} is active.
\end{methoddesc}

\begin{methoddesc}{DribbleOff}{}
Turn off \emph{dribble} and close the \emph{dribble} file.
\end{methoddesc}

\begin{methoddesc}{DribbleOn}{fn}
Enable \emph{dribble} on the file identified by provided filename
\var{fn}.
\end{methoddesc}

\begin{memberdesc}[property]{ExternalTraceback}
Flag to enable or disable printing traceback messages to Python
\var{sys.stderr} if an error occurs when the CLIPS engine calls a
Python function. Please note that the error is not propagated to the
Python interpreter. See the appendices for a more detailed explaination.
\end{memberdesc}

\begin{memberdesc}[property]{FactsWatched}
Flag to enable or disable trace of \class{Fact} assertions and
retractions.
\end{memberdesc}

\begin{memberdesc}[property]{FunctionsWatched}
Flag to enable or disable trace of start and finish of \class{Functions}.
\end{memberdesc}

\begin{memberdesc}[property]{GenericFunctionsWatched}
Flag to enable or disable trace of start and finish of \class{Generic}
functions.
\end{memberdesc}

\begin{memberdesc}[property]{GlobalsWatched}
Flag to enable or disable trace of assignments to \class{Global}
variables.
\end{memberdesc}

\begin{memberdesc}[property]{MethodsWatched}
Flag to enable or disable trace of start and finish of \code{Methods}
within \class{Generic} functions.
\end{memberdesc}

\begin{memberdesc}[property]{MessageHandlersWatched}
Flag to enable or disable trace of start and finish of
\code{MessageHandlers}.
\end{memberdesc}

\begin{memberdesc}[property]{MessagesWatched}
Flag to enable or disable trace of start and finish of \emph{messages}.
\end{memberdesc}

\begin{memberdesc}[property]{RulesWatched}
Flag to enable or disable trace of \class{Rule} firings.
\end{memberdesc}

\begin{memberdesc}[property]{SlotsWatched}
Flag to enable or disable trace of changes to \class{Instance}
\code{Slots}.
\end{memberdesc}

\begin{memberdesc}[property]{StatisticsWatched}
Flag to enable or disable reports about timings, number of \code{facts}
and \code{instances}, and other information after \function{Run()} has
been performed.
\end{memberdesc}

\begin{methoddesc}{UnwatchAll}{}
Turn off \emph{watch} for all items above.
\end{methoddesc}

\begin{methoddesc}{WatchAll}{}
\emph{Watch} all items above.
\end{methoddesc}

\note{Other CLIPS I/O streams besides \var{TraceStream} can be involved
in the trace process: please refer to the CLIPS guides for details.}


\subsection{Memory Operations\label{pyclips-cl-statusconf-memory}}

This object provides access to the memory management utilities of the
underlying CLIPS engine. As said above, it allows the reporting of memory
usage and the attempt to free memory that is used not for computational
purposes. Also, a property of this object affects the engine behaviour
about whether or not to cache some information. Here is what the object
exposes:

\begin{memberdesc}[property]{Conserve}
When set to \constant{True}, the engine does not cache \emph{pretty-print
forms} to memory, thus being more conservative.
\end{memberdesc}

\begin{memberdesc}[property]{EnvironmentErrorsEnabled}
When set to \constant{True}, the engine is enabled to directly write
fatal environment errors to the console (\constant{stderr}). This kind
of messages is in most of the cases printed when the program exits, so
it can be annoying. The behaviour is disabled by default.
\end{memberdesc}

\begin{methoddesc}{Free}{}
Attempt to free as much memory as possible of the one used by the
underlying CLIPS engine for previous computations.
\end{methoddesc}

\begin{memberdesc}[property]{PPBufferSize}
Report the size (in bytes) of the buffers used by \pyclips{} to return
\emph{pretty-print forms} or similar values. By default this is set to
8192, but the user can modify it using values greater than or equal to
256. Greater values than the default can be useful when such forms are
used to reconstruct CLIPS entities and definitions are so complex that
the default buffer size is insufficient.
\end{memberdesc}

\begin{memberdesc}[property]{Requests}
Read-only property reporting the number of memory request made by the
engine to the operating system since \pyclips{} has been initialized.
\end{memberdesc}

\begin{memberdesc}[property]{Used}
Read-only property reporting the amount, in kilobytes, of memory used by
the underlying CLIPS engine.
\end{memberdesc}

\begin{memberdesc}[property]{NumberOfEnvironments}
Read-only property reporting the number of currently allocated
\class{Environment}s.
\end{memberdesc}


\section{I/O Streams\label{pyclips-cl-iostreams}}

In order to be more embeddable, CLIPS defines a clear way to redirect its
messages to locations where they can be easily retrieved. CLIPS users can
access these locations for reading or writing by specifying them as
\emph{logical names} (namely \code{stdin}, \code{stdout}, \code{wtrace},
\code{werror}, \code{wwarning}, \code{wdialog}, \code{wdisplay},
\code{wprompt})\footnote{CLIPS also defines \code{t} as a \emph{logical
name}: as stated in \clipsapg{} this indicates \code{stdin} in functions
that read text and \code{stdout} in function that print out. In
\pyclips{}, for all functions that print out to \code{t} the user must
read from \emph{StdoutStream}.}. \pyclips{} creates some special unique
objects\footnote{\pyclips{} in fact defines one more I/O stream, called
\code{temporary}, which is used internally to retrieve output from CLIPS
that shouldn't go anywhere else. \pyclips{} users however are not supposed
to interact with this object.}, called \emph{I/O streams} throughout this
document, to allow the user to read messages provided by the underlying
engine. Most of these objects have only one method, called \function{Read()},
that consumes CLIPS output and returns it as a string: this string contains
all output since a previous call or module initialization. The only exception
is \var{StdinStream} whose single method is \function{Write()} and it
accepts a string\footnote{The current implementation converts the
argument to a string, so other types can be accepted.} as parameter. As
CLIPS writes line-by-line, the string resulting from a call to
\function{Read()} can contain newline characters, often indicating
subsequent messages.

Here is a list of the \emph{I/O streams} provided by \pyclips{}, along
with a brief description of each.

\begin{tableii}{l|l}{var}{Stream}{Description}
	\lineii{StdoutStream}{where information is usually printed out
            (eg. via \code{(printout t ...)})}
	\lineii{TraceStream}{where trace information (see \emph{watch})
            goes}
	\lineii{ErrorStream}{where CLIPS error messages are written in
            readable form}
	\lineii{WarningStream}{where CLIPS warning messages are written
            in readable form}
	\lineii{DialogStream}{where CLIPS informational messages are
            written in readable form}
	\lineii{DisplayStream}{where CLIPS displays information (eg.
            the output of the \code{(facts)} command)}
	\lineii{PromptStream}{where the CLIPS prompt (normally
            \code{CLIPS>}) is sent}
	\lineii{StdinStream}{where information is usually read by CLIPS
            (eg. via \code{(readline)})}
\end{tableii}

Some of the provided \emph{I/O streams} are actually not so relevant for
\pyclips{} programmers: for instance, it is of little use to read the
contents of \var{PromptStream} and \var{DisplayStream}. In the latter
case, in fact, there are other inspection functions that provide the
same information in a more structured way than text. However they are
available to provide a closer representation of the programming interface
and allow CLIPS programmers to verify if the output of \emph{CLIPS-oriented}
calls (see the paragraph about \function{Build()} and \function{Eval()}
in the appendices) really do what they are expected to.



\section{Predefined \class{Class}es\label{pyclips-cl-stockclasses}}

\pyclips{} defines\footnote{At the module level only: defining these
objects at the \emph{environment} level could cause aliasing current
CLIPS enviroment. On the other hand, if these objects were implemented in
a way that checked for aliasing, access to the actual entities would be
surely slower only favouring compactness of user code.}, some \class{Class}
objects, that is the ones that are present in CLIPS itself by default. They
are defined in order to provide a compact access to CLIPS ``stock'' classes:
most of these objects are of little or no use generally (although they
can be handy when testing for class specification or generalization), but
at least one (\var{USER_CLASS}) can be used to make code more readable.

Namely, these \class{Class}es are:

\begin{tableii}{l|l}{var}{Python Name}{CLIPS defclass}
	\lineii{FLOAT_CLASS}{FLOAT}
	\lineii{INTEGER_CLASS}{INTEGER}
	\lineii{SYMBOL_CLASS}{SYMBOL}
	\lineii{STRING_CLASS}{STRING}
	\lineii{MULTIFIELD_CLASS}{MULTIFIELD}
	\lineii{EXTERNAL_ADDRESS_CLASS}{EXTERNAL-ADDRESS}
	\lineii{FACT_ADDRESS_CLASS}{FACT-ADDRESS}
	\lineii{INSTANCE_ADDRESS_CLASS}{INSTANCE-ADDRESS}
	\lineii{INSTANCE_NAME_CLASS}{INSTANCE-NAME}
	\lineii{OBJECT_CLASS}{OBJECT}
	\lineii{PRIMITIVE_CLASS}{PRIMITIVE}
	\lineii{NUMBER_CLASS}{NUMBER}
	\lineii{LEXEME_CLASS}{LEXEME}
	\lineii{ADDRESS_CLASS}{ADDRESS}
	\lineii{INSTANCE_CLASS}{INSTANCE}
	\lineii{INITIAL_OBJECT_CLASS}{INITIAL-OBJECT}
	\lineii{USER_CLASS}{USER}
\end{tableii}

The following code, shows how to use the ``traditional''
\function{BuildClass()} factory function and how to directly subclass one
of the predefined \class{Class} object. In the latter case, probably, the
action of subclassing is expressed in a more clear way:

\begin{verbatim}
>>> import clips
>>> C = clips.BuildClass("C", "(is-a USER)(slot s)")
>>> print C.PPForm()
(defclass MAIN::C
   (is-a USER)
   (slot s))

>>> D = clips.USER_CLASS.BuildSubclass("D", "(slot s)")
>>> print D.PPForm()
(defclass MAIN::D
   (is-a USER)
   (slot s))
\end{verbatim}

Although it actually does not save typing (the statement is slightly
longer), the second form can be used to produce more readable Python code.

\note{These objects are actually useful \emph{only} when the package is
fully imported, that is using the \code{import clips} form: importing
the symbols at global level (in the form \code{from clips import *}) does
in fact create some namespace problems. Since in the latter case the names
that represent stock classes are only references to the ones defined at
module level, \pyclips{} cannot change them when the actual classes are
relocated in the CLIPS memory space, for instance when \function{Clear}
is called.}
		% Provided Objects
% $Id: appendix.tex 346 2008-02-25 00:39:00Z Franz $
\appendix

\chapter{Usage Notes\label{pyclips-unotes}}

\section{Environments\label{pyclips-unotes-env}}

As seen in the detailed documentation, \pyclips{} also provides access
to the \emph{environment} features of CLIPS, through the
\class{Environment} class. \class{Environment} objects provide almost
all functions normally available at the top level of \pyclips{}, that is
importing \code{clips} into a Python script. \class{Environment} object
methods having the same name of functions found at the top level of
\pyclips{}, have the same effect of the corresponding function -- but
restricted to the logical environment represented by the object
itself\footnote{In fact, the Python submodule that implements the
\class{Environment} class is generated automatically: the process can be
examined by looking at the code in \file{setup.py} and
\file{clips/_clips_wrap.py}.}. Normally in a true CLIPS session users would
not use environments, so the concept of environment may, in many cases,
not be useful.

There are also functions (namely: \function{CurrentEnvironment()} and
\function{Environment.SetCurrent()}) that allow the user to switch
environment and to use top level functions and classes in any of the
created environments. This is useful in cases where environments are used,
because due to the double nature of CLIPS API (see \clipsapg{} about
\emph{companion functions} for standard API), objects defined in
environments have slightly different types than corresponding top level
objects -- since these types are \emph{environment-aware}. However
environment aware classes define exactly the same methods as the top
level counterparts, so their logical use is the same.

\begin{notice}
Please note that the \function{CurrentEnvironment()} function returns
a fully functional \class{Environment} object. However the system
prevents\footnote{Raising an appropriate exception.} invasive access via
\emph{environment-aware} functions to current \emph{environment}: user
code should \emph{always} use functions defined at module level to access
current \emph{environment}. The \class{Environment} methods become safe
as soon as another \class{Environment} has become current -- and in this
case its methods and properties will raise an error.
\end{notice}

\class{Environment}s are a limited resource: this is because it is
impossible in \pyclips{} to destroy a created \class{Environment}.
In order to reuse \class{Environment}s it may be useful to keep a
reference to each of them for the whole \pyclips{} session. If there
is an attempt to create more \class{Environment}s than allowed, an
exception is raised.



\section{Multiple Ways\label{pyclips-unotes-multiw}}

There is more than one way to use the \pyclips{} module, since it exposes
almost all the API functions of CLIPS, seen in fact as a library, to the
Python environment.

The module \pyclips{} provides some functions, that is \function{Build()}
and \function{Eval()}, that let the user directly issue commands in the
CLIPS subsystem from a Python program. In fact, the use of \function{Build()}
allows definition of constructs in CLIPS\footnote{Note that the
\function{Build()} function does not return any value or object, so you
will have to call \function{Find\emph{Construct}()} to find entities created
using the \function{Build()} function.}, and \function{Eval()} lets the user
evaluate values or call code directly in the subsystem. So for instance
rules can be built in \pyclips{} using

\begin{verbatim}
>>> import clips
>>> clips.Build("""
(defrule duck-rule "the Duck Rule"
   (duck)
   =>
   (assert (quack)))
""")
>>> clips.PrintRules()
MAIN:
duck-rule
\end{verbatim}

and evaluate a sum using

\begin{verbatim}
>>> n = clips.Eval("(+ 5 2)")
>>> print n
7
\end{verbatim}

Also, the user is allowed to call functions that do not return a value
using \function{Eval()}\footnote{There is a discussion about functions
that only have \emph{side effects} in CLIPS, such as \code{printout}, in
\clipstut{}, that is, the CLIPS tutorial.}, as in the following example:

\begin{verbatim}
>>> clips.Eval('(printout t "Hello, World!" crlf)')
>>> print clips.StdoutStream.Read()
Hello, World!
\end{verbatim}

There is another function, namely \function{SendCommand()}, that sends
an entire CLIPS command (it has to be a full, correct command, otherwise
\pyclips{} will issue an exception): as \function{Build()} it does not
return any value or object\footnote{Some information about the command
result can be retrieved reading the appropriate output streams.}.
However this can be particularly useful when the user needs to implement
an interactive CLIPS shell within an application built on \pyclips{}.
Unless the application is mostly CLIPS oriented (if for instance Python
is used just as a ``glue'' script language) probably the use of this
function has to be discouraged, in favour of the code readability that
-- at least for Python programmers -- is provided by the Python oriented
interface.

Using \function{SendCommand()} it becomes possible to write:

\begin{verbatim}
>>> import clips
>>> clips.SendCommand("""
(defrule duck-rule "the Duck Rule"
   (duck)
   =>
   (assert (quack)))
""")
>>> clips.SendCommand("(assert (duck))")
>>> clips.Run()
>>> clips.PrintFacts()
f-0     (duck)
f-1     (quack)
For a total of 2 facts.
>>> clips.PrintRules()
MAIN:
duck-rule
\end{verbatim}

The most important caveat about \function{SendCommand()} is that CLIPS
accepts some kinds of input which normally have to be considered
incorrect, and \pyclips{} does neither return an error value, nor raise
an exception: for instance, it is possible to pass a symbol to CLIPS to
the command line as in

\begin{verbatim}
CLIPS> thing
thing
\end{verbatim}

and in this case CLIPS ``evaluates'' the symbol, printing it to the
console as a result of the evaluation. \pyclips{} does not automatically
capture evaluation output, and just accepts a symbol (or other commands
that can be evaluated) as input without any production:

\begin{verbatim}
>>> import clips
>>> clips.SendCommand("thing")
\end{verbatim}

but, turning on the verbosity flag:

\begin{verbatim}
>>> clips.SendCommand("thing", True)
>>> print clips.StdoutStream.Read()
thing
\end{verbatim}

Of course, \pyclips{} complains if something incorrect is passed to the
\function{SendCommand()} function and raises an exception as previously
stated. However the exception is accompanied by a rather non-explanatory
text. The \var{ErrorStream} object should be queried in this case in order
to obtain some more information about the error:

\begin{verbatim}
>>> clips.SendCommand("(assert (duck)")	# no right bracket

Traceback (most recent call last):
  File "<pyshell#5>", line 1, in -toplevel-
    clips.SendCommand("(assert (duck)")	# no right bracket
  File ".../_clips_wrap.py", line 2575, in SendCommand
    _c.sendCommand(s)
ClipsError: C09: unable to understand argument
>>> print clips.ErrorStream.Read()

[PRNTUTIL2] Syntax Error:  Check appropriate syntax for RHS patterns.
\end{verbatim}

Obviously \function{SendCommand()} can lead to serious errors if not
used with some kind of interaction.

The point of this paragraph is that, for entity definition (evaluation
can only be performed using the \function{Eval()} or \function{Call()}
functions), the \pyclips{} module provides a full set of specific
\function{Build\emph{Entity}()} functions which also return appropriate
objects corresponding to specific entities. So, the task of building a
\emph{rule} in CLIPS (in fact, a \class{Rule} object in Python) could
preferably be performed directly using the \function{BuldRule()}
function, that is:

\begin{verbatim}
>>> clips.Clear()
>>> clips.Reset()
>>> r0 = clips.BuildRule("duck-rule", "(duck)", "(assert (quack))",
                         "the Duck Rule")
>>> print r0.PPForm()
(defrule MAIN::duck-rule "the Duck Rule"
   (duck)
   =>
   (assert (quack)))

>>> clips.PrintRules()
MAIN:
duck-rule
\end{verbatim}

thus with the same effect as with the \function{Build()} function, but
obtaining immediately a reference to the rule entity in CLIPS as a Python
object. Similar examples could be provided for the \function{SendCommand()}
function, using the appropriate constructs or commands that can be used
to achieve the same goals.

This allows the user to choose between at least two programming styles in
\pyclips{}: the former, more CLIPS oriented, relies heavily on the use of
the \function{Build()}, \function{Eval()} and \function{SendCommand()}
functions, and is probably more readable to CLIPS developers. The latter
is somewhat closer to Python programming style, based on the creation of
objects of a certain nature by calling specific Python functions. The
advice is to avoid mixing the two styles unless necessary, since it can
make the code quite difficult to understand.



\section{Python Functions in CLIPS\label{pyclips-unotes-pyfuncs}}

In \pyclips{} it is possible to execute Python functions from within
CLIPS embedded constructs and statements. This allows the extension of
the underlying inference engine with imperative functionalities, as well
as the possibility to retrieve information from the Python layer
asynchronously with respect to Python execution. Of course this
possibility enables some enhancements of the CLIPS environment, but
-- as a drawback -- it also opens the way to errors and misunderstandings.

Usage of Python external functions is fairly simple: the user should
register functions that will be called from within the CLIPS subsystem
in \pyclips{} using the \function{RegisterPythonFunction()} toplevel
function. If no alternate name for the function is specified, then the
Python name will be used\footnote{An example of function registration
has been provided in the introduction.}. If necessary, Python function
names can be deregistered using \function{UnregisterPythonFunction()} and
\function{ClearPythonFunctions()} utilities. Once a function is registered
it can be called from within the CLIPS engine using the following syntax:

\begin{verbatim}
    (python-call <funcname> [arg1 [arg2 [ ... [argN]]]])
\end{verbatim}

and will return a value (this allows its use in assignments) to the CLIPS
calling statement. In the call, \code{<funcname>} is a \emph{symbol}
(using a string will result in an error) and the number and type of
arguments depends on the actual Python function. When arguments are of
wrong type or number, the called function fails. Using the previously
illustrated \function{py_square} example, we have:

\begin{verbatim}
>>> clips.RegisterPythonFunction(py_square)
>>> clips.SetExternalTraceback(True)	# print traceback on error
>>> print clips.Eval("(python-call py_square 7)")
49
>>> print clips.Eval('(python-call py_square "a")')
Traceback (most recent call last):
  File ".../_clips_wrap.py", line 2702, in <lambda>
    f = lambda *args: _extcall_retval(func(*tuple(map(_cl2py, list(args)))))
  File "<pyshell#85>", line 2, in py_square
TypeError: can't multiply sequence to non-int
FALSE
>>> print clips.Eval("(python-call py_square 7 7)")
Traceback (most recent call last):
  File ".../_clips_wrap.py", line 2702, in <lambda>
    f = lambda *args: _extcall_retval(func(*tuple(map(_cl2py, list(args)))))
TypeError: py_square() takes exactly 1 argument (2 given)
FALSE
\end{verbatim}

It is important to know, in order to avoid errors, that the Python
interpreter that executes functions from within CLIPS is exactly the
same that calls the \pyclips{} function used to invoke the engine: this
means, for example, that a Python function called in CLIPS is subject
to change the state of the Python interpreter itself. Moreover, due
to the nature of CLIPS external function call interface, Python functions
called in CLIPS will never raise exceptions\footnote{Exceptions can
arise \emph{during} the Python function execution, and can be caught
inside the function code. However, for debugging purposes, there is the
possibility to force \pyclips{} print a standard traceback whenever an
error occurs in a Python function called by CLIPS.} in the Python calling
layer.

Here are some other issues and features about the nature of the external
function call interface provided by \pyclips{}:

\emph{Functions should be CLIPS-aware:} when CLIPS calls a Python external
function with arguments, these are converted to values that Python can
understand using the previously described \emph{wrapper classes}. Thus,
for instance, if the Python function is given an integer argument, then
an argument of type \class{Integer} (not \class{int}) will be passed as
actual parameter. This means that in most cases \pyclips{} has to be
imported by modules that define external Python functions.

\emph{Actual parameters cannot be modified:} there is no way to pass values
back to CLIPS by modifying actual parameters. The possibility to use
\class{Multifield} parameters as lists should not deceive the user, as
every modification performed on \class{Multifield}s that Python receives
as parameters will be lost after function completion. A way to handle this
is to treat parameters as \emph{immutable} values.

\emph{External functions should always return a value:} functions always
return a value in CLIPS, even in case of an error. This can be clearly
seen in the following chunk of CLIPS code:

\begin{verbatim}
CLIPS> (div 1 0)
[PRNTUTIL7] Attempt to divide by zero in div function.
1
\end{verbatim}

where, although an error message is printed to the console, the value
\constant{1} is returned by the system. In the same way, CLIPS expects
Python external functions to return a value. \pyclips{} solves this issue
by converting a return value of \constant{None} (which is the real return
value for Python functions that simply \code{return}) into the symbol
\code{nil}, that has a meaning similar to the one of \constant{None} for
Python. Also, functions that raise uncaught exceptions will in fact return
a value to the underlying CLIPS engine: in this case the returned value
is the symbol \code{FALSE}, and an error message is routed to the error
stream -- thus, it can be retrieved using \function{ErrorStream.Read()}.
The following example imitates the \code{div} CLIPS example above:

\begin{verbatim}
>>> import clips
>>> exceptor = lambda : 1 / 0
>>> clips.RegisterPythonFunction(exceptor, 'exceptor')
>>> clips.SetExternalTraceback(True)	# print traceback on error
>>> clips.Eval('(python-call exceptor)')
Traceback (most recent call last):
  File ".../_clips_wrap.py", line 2702, in <lambda>
    f = lambda *args: _extcall_retval(func(*tuple(map(_cl2py, list(args)))))
  File "<pyshell#79>", line 1, in <lambda>
ZeroDivisionError: integer division or modulo by zero
<Symbol 'FALSE'>
\end{verbatim}

\emph{Return values must be understood by CLIPS:} only values that can
be converted to CLIPS base types can be returned to the inference engine.
This includes all values that can be converted to \pyclips{} \emph{wrapper
classes}. In fact it can be considered a good practice to cast return
values to \pyclips{} \emph{wrapper classes} when the main purpose of a
function is to be called from within CLIPS.

\emph{Python functions act as generic functions:} due to the nature of
Python, functions are generally polymorphic:

\begin{verbatim}
>>> def addf(a, b):
        return a + b
>>> print addf("egg", "spam")
eggspam
>>> print addf(2, 4)
6
\end{verbatim}

The intrinsic polymorphism of Python functions is kept within the CLIPS
subsystem:

\begin{verbatim}
>>> import clips
>>> clips.RegisterPythonFunction(addf)
>>> print clips.Eval('(python-call addf "egg" "spam")')
eggspam
>>> print clips.Eval('(python-call addf 2 4)')
6
\end{verbatim}

Thus Python functions act in a way that is similar to \code{generic}s.



\chapter{The \module{clips._clips} Submodule\label{pyclips-llclips}}

It has been said throughout the whole document that there are two
different ``natures'' of \pyclips{}, called the \emph{high level module}
and the \emph{low level module}. The former has been described in detail
here, and is supposed to be the main interface to CLIPS. However, since
all communication between Python and the CLIPS subsystem is implemented
in the \emph{low level} part, some users might find it useful to access
this interface instead of the \emph{higher level} one. It is not the
intention of this manual to provide documentation for the \emph{low
level} \module{clips._clips} interface, but only to give some indications
to users who already have experience of the CLIPS C interface.

Submodule \module{clips._clips} provides low-level classes that have the
same names as their counterparts in CLIPS, such as \class{fact},
\class{deftemplate} and so on. Also, \module{clips._clips} defines an
\class{environment} class to refer to \emph{environment} addresses.

Almost all functions described in \clipsapg{} are ported to Python,
except for missing features described below: the name is the same as in
the reference guide, except for the first letter that is not
capitalized\footnote{There can be some exceptions: the most
important one is the assertion function, since Python already has an
\keyword{assert} keyword. This function is called \function{assertFact}
instead.}. CLIPS ``top level'' functions have been ported to
\module{clips._clips} as well as \emph{companion functions} that accept
an \class{environment} instance as the first argument\footnote{When
trying to show the \emph{documentation string} of these functions, the
first argument is not described because their code has been generated
automatically.}, whose name begins with \code{env_} followed by the same
name as the corresponding top level function.

\emph{Low level} functions are documented by themselves through
\emph{documentation strings}, that describe purpose, arguments and return
values. For instance, let's show the documentation of a function:

\begin{verbatim}
>>> import clips
>>> cl = clips._clips
>>> print cl.listDeftemplates.__doc__
listDeftemplates(logicalname [, module])
list deftemplates to output identified by logicalname
arguments:
  logicalname (str) - the logical name of output
  module (module) - the module to inspect, all modules if omitted
\end{verbatim}

Most low level function documentation strings, i.e. the ones given
for functions that are not trivially identifiable with the CLIPS API
counterparts\footnote{Some functions have a documentation string that
actually refers to the CLIPS API itself, explicitly containing the words
``\emph{equivalent of C API}'' and the C function name: \clipsapg{} is
especially useful in this case.}, have this form and describe in detail
arguments and return values. Users who want to benefit of these
functions, can display \emph{documentation strings} as a reference.

The underlying engine is the same, there is no separation of environment
between the two interfaces. Operations performed at \emph{lower level}
are obviously reflected in the \emph{higher level} layer, as in the
following example:

\begin{verbatim}
>>> clips.Clear()
>>> cl.facts("stdout")
>>> s = clips.StdoutStream.Read()
>>> print s
None
>>> print cl.assertString.__doc__
assertString(expr) -> fact
assert a fact into the system fact list
returns: a pointer to the asserted fact
arguments:
  expr (str) - string containing a list of primitive datatypes
>>> f = cl.assertString("(duck)")
>>> clips.PrintFacts()
f-0     (duck)
For a total of 1 fact.
\end{verbatim}

so the two interfaces can be used interchangeably.



\chapter{Error Codes\label{pyclips-errors}}

It has been discussed above, that some of the \pyclips{} functions can
raise a CLIPS specific exception, namely \exception{ClipsError}. Some
of the exceptions of this type (in fact, the ones raised by the
underlying CLIPS engine and caught at the lower level), come with an
error code in the accompanying text. A brief description of exceptions
that arise at low level follows:

\begin{tableii}{l|l}{code}{Code}{Description}
	\lineii{P01}{An object could not be created, due to memory
            issues}
	\lineii{C01}{The engine could not create a system object}
	\lineii{C02}{The referred object could not be found in
            subsystem}
	\lineii{C03}{An attempt was made to modify an unmodifiable
            object}
	\lineii{C04}{A file could not be opened}
	\lineii{C05}{The current environment could not be retrieved}
	\lineii{C06}{The engine was unable to return the required value}
	\lineii{C07}{Parse error in the passed in CLIPS file}
	\lineii{C08}{Syntax or parse error in the passed in CLIPS
            expression}
	\lineii{C09}{Syntax or parse error in the passed in argument}
	\lineii{C10}{Expression could not be evaluated, maybe because of
            syntax errors}
	\lineii{C11}{An object could not be removed from the CLIPS
            subsystem}
	\lineii{C12}{A fact could not be asserted, maybe because of
            missing \code{deftemplate}}
	\lineii{C13}{Iteration beyond last element in a list}
	\lineii{C14}{A CLIPS function has been called unsuccessfully}
	\lineii{C15}{Attempt to modify an already asserted fact was made}
	\lineii{C16}{Cannot destroy environment while it is current}
	\lineii{C90}{Other errors: specific description given}
	\lineii{C97}{The feature is present when a higher version of
            CLIPS is used}
	\lineii{C98}{An attempt was made to use an unimplemented feature
            (see below)}
	\lineii{C99}{Generic CLIPS error, last operation could not be
            performed}
	\lineii{C00}{Generic CLIPS error, no specific cause could be
            reported}
        \lineii{S01}{Attempt to access a fact that is no longer valid
            or has been deleted}
        \lineii{S02}{Attempt to access an instance that is no longer
            valid or has been deleted}
        \lineii{S03}{Clear operation on an environment has failed}
        \lineii{S04}{Attempt to access an environment that has been
            deleted}
        \lineii{S05}{Attempt to operate on alias of current environment}
        \lineii{S06}{Attempt to create more environments than allowed}
        \lineii{S00}{Generic internal system error}
        \lineii{R01}{The logical buffer (I/O Stream) has been misused}
	\lineii{R02}{The logical buffer with given name could not be
            found}
	\lineii{R03}{Could not write to a read-only logical buffer}
\end{tableii}

These codes can be extracted from the exception description and used to
determine errors -- for instance, in an \code{if ... elif ...} control
statement. Some of these errors are caught by the \pyclips{} high-level
layer and interpreted in different ways (e.g. the \code{C13} error is
used to generate lists or to return \constant{None} after last element
in a list).

There are also some CLIPS specific exceptions that can be thrown at the
higher level: they are identified by a code beginning with the letter
\code{M}. A list of these exceptions follows, along with their
description:

\begin{tableii}{l|l}{code}{Code}{Description}
        \lineii{M01}{A constructor could not create an object}
        \lineii{M02}{An object could not be found in the CLIPS subsystem}
        \lineii{M03}{An attempt was made to \function{pickle} an object}
        \lineii{M99}{Wrong Python version, module could not initialize}
\end{tableii}

Finally, there is a particular error that occurs in case of \emph{fatal}
memory allocation failures, which is identified by a particular
exception, namely \exception{ClipsMemoryError}. This excepion is raised
with the following code and has the following meaning:

\begin{tableii}{l|l}{code}{Code}{Description}
        \lineii{X01}{The CLIPS subsystem could not allocate memory}
\end{tableii}

In this case the calling program \emph{must} exit, as the underlying
engine has reached an unstable state. An exception different from the
standard \exception{ClipsError} has been provided in order to allow
quick and effective countermeasures\footnote{Please note that in some
cases, and depending on how the operating system treats memory
allocation failures, the Python interpreter too could loose stability
in case of memory shortage.}.



\chapter{Multithreading\label{pyclips-threading}}

The CLIPS engine is a separate subsystem, as stated many times before.
In other words, it maintains a state independently from what happens in
the Python interpreter. This also means that, since CLIPS was never
conceived to be used as a multithreaded library, multiple threads should
not try to access the engine concurrently. The Python interpreter, due
to its nature, does not actually allow concurrent calls to the low-level
module, so it is safe to create concurrent threads that interact with
\pyclips{}.

However this has the side effect that, during a time consuming task
(such as calling \function{Run()} on a complex set of rules and facts)
the calling thread may block the other ones.

A partial solution to this, to allow multiple threads to switch more
reactively, is to call \function{Run()} with the \var{limit} parameter,
which specifies the number of rules to be fired at once. Of course this
allows subsequent calls to the CLIPS engine to modify its state, and
consequently enables execution paths that could be different from the
ones that a full CLIPS \code{(run)} would normally cause. Obviously
this only applies to the \function{Run()} function.

The consideration also implies that multithreaded Python applications
should take care of examining the engine state before interacting with
it, especially when splitting \function{Run()} (which normally modifies
the state) in multiple calls.



\chapter{Missing Features\label{pyclips-missing}}

Most of the CLIPS API is implemented in \pyclips{}. The \emph{lower
level} interface (which directly maps CLIPS exposed functions as
described in \clipsapg{}) can be accessed using the
\module{clips._clips} submodule. Almost all the functions defined here
have a counterpart in the CLIPS API, and a combined use of documentation
strings and \clipsapg{} itself can allow you to directly manipulate the
CLIPS subsystem as you would have done by embedding it in a C program.
However, there are some functions that appear in \code{dir(clips._clips)}
but, when called, issue an error:

\begin{verbatim}
>>> clips._clips.addClearFunction()
Traceback (most recent call last):
  File "<pyshell#46>", line 1, in ?
    clips._clips.addClearFunction()
ClipsError: C98: unimplemented feature/function
\end{verbatim}

and in fact, even their documentation string reports

\begin{verbatim}
>>> print clips._clips.addClearFunction.__doc__
unimplemented feature/function
\end{verbatim}

even if the name of the function is defined and is a \emph{callable}.

A list of such functions follows:

\begin{tableii}{l|l}{code}{Function}{Type}
	\lineii{addClearFunction}{execution hook}
	\lineii{addPeriodicFunction}{execution hook}
	\lineii{addResetFunction}{execution hook}
	\lineii{removeClearFunction}{execution hook}
	\lineii{removePeriodicFunction}{execution hook}
	\lineii{removeResetFunction}{execution hook}
	\lineii{addRunFunction}{execution hook}
	\lineii{removeRunFunction}{execution hook}
	\lineii{decrementFactCount}{reference count handler}
	\lineii{incrementFactCount}{reference count handler}
	\lineii{decrementInstanceCount}{reference count handler}
	\lineii{incrementInstanceCount}{reference count handler}
	\lineii{setOutOfMemoryFunction}{memory handler hook}
	\lineii{addEnvironmentCleanupFunction}{execution hook}
	\lineii{allocateEnvironmentData}{memory handler}
	\lineii{deallocateEnvironmentData}{memory handler}
	\lineii{destroyEnvironment}{environment destructor}
	\lineii{getEnvironmentData}{memory handler}
\end{tableii}

The description of these functions is outside the scope of this guide,
and can be found in \clipsapg{} of CLIPS. It is not likely that these
functions will be implemented even in future versions of \pyclips{} since
Python programmers are usually not interested in dealing with low level
memory handling (which is the primary use of the memory oriented
functions), and tasks like reference count handling are performed
directly by Python itself (for Python objects which shadow CLIPS
entities) and by the low level \pyclips{} submodule. Also, the functions
referred to above as \emph{execution hooks} often have to deal with CLIPS
internal structures at a very low level, so they would be of no use in a
Python program.

Other API functions, which are used internally by \pyclips{} (for
instance the \function{InitializeEnvironment()} C function), are not
implemented in the module.

\begin{notice}
Some of the features (either in current and possibly in further versions
of \pyclips{}) may depend on the version of CLIPS that is used to compile
the module. Using the most recent stable version of CLIPS is recommended
in order to enable all \pyclips{} features. Features that are excluded
from the module because of this reason will issue an exception, in which
the exception text reports the following: \code{"C97: higher engine
version required"}. Moreover, the CLIPS engine version may affect the
behaviour of some functions. Please consider reading the documentation
related to the used CLIPS version when a function does not behave as
expected.
\end{notice}


\chapter{Installing \pyclips{}\label{pyclips-setup}}

\section{Installation\label{pyclips-setup-installation}}

To install \pyclips{} you should also download the full CLIPS source
distribution. You will find a file called \file{CLIPSsrc.zip} at the
CLIPS download location: you should choose to download this instead of
the \UNIX{} compressed source, since the setup program itself performs
the task of extracting the files to an appropriate directory with the
correct line endings. The ZIP file format has been chosen in order to
avoid using different extraction methods depending on the host operating
system.

\pyclips{} uses \module{distutils} or \module{setuptools} for its
installation. So in all supported systems the module can be easily
set up once the whole source has been extracted to a directory and
the CLIPS source code has been put in the same place, by means of the
following command:

\begin{verbatim}
# python setup.py install
\end{verbatim}

In fact recent versions of \pyclips{} will attempt to download the
latest supported CLIPS source directly from the \pyclips{} web site
if no CLIPS source package is found. Otherwise no attempt to connect
to the Internet will be made. The \file{README} file provides more
up-to-date and detailed information on the setup process.

On \UNIX{}, if you have a system-wide Python distribution, your
privileges for installation should be the same as the Python owner.

The CLIPS library itself is compiled for a specific platform, since
\file{setup.py} modifies the \file{setup.h} file in the CLIPS
distribution. 

\pyclips{} is known to build and pass the tests on \emph{Linux (x86 and
x86_64)}\footnote{The x86_64 platforms requires some optional patches
to be applied.}, \emph{Win32} (many flavours of it have been tested),
\emph{Sun Solaris} with 32-bit gcc, \emph{FreeBSD}, \emph{Mac OS X} with
\emph{Fink} and, using a customized build process, has been ported to the
\emph{Sharp Zaurus (SA-1110)} platform.


\section{Requirements\label{pyclips-setup-requirements}}

\pyclips{} requires Python 2.4 or above to function: it uses decorators
to check and enforce types where needed, and in some places it also uses
modern aspects of the Python API.

At least version 6.23 of CLIPS is required: it allows the definition and
use of \emph{environments}, and the function and macro definitions are
more conformant to the ones described in \clipsapg{}. Of course features
present in CLIPS 6.24 are not available when using the previous CLIPS
version, so if there is no particular reason to use it, please compile
PyCLIPS with CLIPS 6.24, which also fixes some bugs.

        % Appendix and Notes

% $Id: license.tex 174 2004-10-16 12:43:58Z Franz $
\chapter{License Information\label{pyclips-license}}

The following is the license text, which you can obtain by issuing a

\begin{verbatim}
>>> import clips
>>> print clips.license
\end{verbatim}

at the Python prompt once the \pyclips{} module has been installed.

\verbatiminput{license.txt}

	        % License Information

\end{document}
