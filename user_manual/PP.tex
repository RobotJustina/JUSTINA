\documentclass[user_manual.tex]{subfiles}
\begin{document}
 \chapter{Primeros pasos}
 Antes de poner en funcionamiento a Justina, debemos conocer los software con los que trabaja y los requerimientos para 
 su correcto funcionamiento.\\
 \\
 Como primer paso debemos conocer el software necesario para el funcionamiento de Justina. 
 
 \section{Software necesario}

Se requiere lo siguiente:
\begin{itemize}
\item Ubuntu 14.04.1 (This is the tested version)
\item ROS Indigo desktop full
\item OpenNI + PrimeSense drivers
\item OpenCV 2.4.8 or 2.4.9. Compiled with OpenNi, WITHOUT OpenCL, WITHOUT Cuda, with Eigen
\item PCL 1.6
\end{itemize}

Para conocer la forma de instalar ROS, OpenNI, los drivers PrimeSense y OpenCV 2.4.9 por favor acude al apéndice B (software).

 \newpage
 Como siguiente paso obtener el software de Justina, para esto debemos descargar todas las carpetas con las que se ha 
 trabajado Justina.
 \section{Obtención de la carpeta de Git hub}
 Todos los repositorios del software de Justina se encuentran en Git hub (así como este manual y es de donde podrás descargar
 futuras versiones). Existen dos formas para obtener la carpeta contenedora con todo lo necesario para empezar:\\
 \\
 La primera es ir a la dirección "https://github.com/RobotJustina/JUSTINA/tree/develop" y descargarlo con el botón color verde
 que dice "clone or download".\\
 \\
 te saldrá una opción para seleccionar la ubicación en la que deseas guardar el archivo .zip\\
 \\
 Busca la carpeta contenedora y descomprime el archivo. Al descomprimirlo obtendrás una carpeta llamada "JUSTINA-develop" la 
 cual contiene todo lo necesario para utilizar a Justina.
 
 La segunda opción
 
\section{Instalación completa del software de Justina}
Una vez instalado ROS procedemos a instalar el software de Justina, para esto abrir una terminal y seguir las siguientes instrucciones.

\begin{enumerate}
 \item meterse a la carpeta JUSTINA-develop y ejecutar JustinaSetup.sh
 \item Aceptar cada que pregunte. Esto nos llevara varios minutos.
 \item Una vez instalado el software, debemos habilitar el uso de los puertos USB para ROS, para esto nos metemos al directorio "JUSTINA-develop/ToInstall
 /USB (si quieres seguir las instrucciones más detalladamente, en la misma dirección abrir el archivo ``instructions'') una vez dentro
 de la carpeta ejecutar el siguiente comando "sudo cp 80-justinaRobot.rules /etc/udev/rules.d/"
 \item Te pedirá la contraseña. Una vez termines de ejecutar el comando, se debe ejecutar el siguiente: %``sudo udevadm control --reload-rules && sudo
      %service udev restart && sudo udevadm trigger''
\end{enumerate}
Listo, ya tienes instalado el software de Justina.

\section{Cómo compilar los repositorios de Justina}
Para compilar los repositorios de Justina simplemente ve al directorio "JUSTINA-develop/catkin\_ws", en este directorio ejecutamos el siguiente
comando "catkin\_make". Esto nos llevara varios minutos.\\
\\
Una vez compilados todos los repositorios ejecutar el siguiente comando dentro de la misma carpeta "source devel/setup.bash".\\
\\
Listo, ahora el software de Justina está instalado y los repositorios compilados y listos para usarse.

\section{RViz y GUI de Justina}
Para probar el funcionamiento del hardware y software de Justina utilizaremos RViz y la GUI. Para ejecutar estos programas utilizamos el 
comando "roslaunch surge\_et\_ambula justina.launch".
\section{Simulación en el RViz y GUI de Justina}
Cuando no tenemos conectado el robot Justina a nuestras laptops lo único que podemos hacer
es simular a Justina en nuestras laptops, para esto ejecutamos el siguiente código "roslaunch surge\_et\_ambula justina\_simul.launch".




\end{document}
