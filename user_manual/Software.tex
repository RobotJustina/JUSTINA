\documentclass[user_manual.tex]{subfiles}
\begin{document}
 
 \chapter{Arquitectura del software}

\section{ViRbot}
El sistema VIRBOT consiste de varios subsistemas los cuales controlan la operación del robot móvil.

\section{Guía de desarrollo}
\begin{itemize}
 \item Todo código fuente DEBE estar contenido en el folder \textit{catkin\_ws/src}.\\
 
 \item Sólo el código contenido en la carpeta \textit{catkin\_ws/src/hardware} puede interactuar con el 
 hardware del robot\\
 
 \item El punto anterior implica que todos los otros programas deberán implementar SÓLO algoritmos. 
 Todas las interacciones con el hardware (e.g.. obtener una imagen desde la cámara, leer el , 
 mover la base o la cabeza, hablar, etc.) debe hacerse intercambiando información con los paquetes
 contenidos en la carpeta \textbf{hardware}, a través de los tópicos y servicios de ROS.
 
 \item Los códigos contenidos en todas las carpetas dentro de \textit{catkin\_ws/src}, excepto las carpetas de 
 herramientas, DEBEN contener sólo código escrito por el propio desarrollador (de cualquier paquete).
 Todas las bibliotecas necesarias o código de otras fuentes (bibliotecas serial, arduino, julius,
 dynamixel, etc.), si no están instaladas en algún default path (/opt/ros, /usr/local/, etc.), deben 
 ser puestas dentro de la carpeta \textit{catkin\_ws/src/tools} en una subcarpeta apropiada.\\
 
 \item Los desarrolladores deben tratar de usar sólo mensajes ya definidos en algún paquete de ROS o
 pila, sin embargo, si mensajes personalizados son requeridos, éstos deben ser puestos dentro de
 \textit{catkin\_ws/src/subsystem/subsystem\_maga}, así que, muchos mensajes pueden ser usados sin necesidad
 de ejecutar todos los demás subsistemas.\\
\end{itemize}

\section{Árbol de carpetas}
\begin{verbatim}
catkin_ws
  build
  devel
  src
      hardware
	  arms
	  battery
	  hardware_state
	  justina_urdf
	  hardware_msgs
	  head
	  mobile_base
	  point_cloud_manager
	  speakers
	  torso
      hri
	  gesture_recog
	  hri_msgs
	  justina_gui
	  natural_language
	  speech_recog
      interoperation
	  bbros_bridge
	  joy_teleop
	  pc_teleop
	  roah_rsbb
      manipulation
	  arms_predef_movs
	  arms_path_planning
	  arms_trajectory_planning
	  head_predef_movs
	  head_tracking_point
	  manipulation_msgs
      navigation
	  localization
	  mapping
	  moving
	  navigation_msgs
	  path_planning
	  point_traking
      planning
	  planning_msgs
	  pomdp
	  rule_based
	  semantic_database
	  state_machines
      surge_et_ambula
	  launch
	  rviz_files
      testing
	  any_not_stable_node
      tools
	  ros_tools
	  libraries
	      serial_arduino
	      serial_dynamixel
	      julius
	      festival
      vision
	  door_detector
	  furniture_recog
	  object_detector
	  object_recog
	  person_detection
	  person_recog
	  vision_msgs
user_manual
\end{verbatim}
%\chapter{Ayuda y referencias}
Cada paquete en la carpeta de \textit{hardware} debe tener su versión simulada, así que, el resto del 
software (todas las otras carpetas se supone que contienen sólo algoritmos y no interacción con el 
hardware del robot) puedes correr inmediatamente el modo de simulación. Eligiendo entre simulado o 
real debe ser hecho en la carpeta de ejecución.


\end{document}
