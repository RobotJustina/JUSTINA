\documentclass[a4paper,usenames,dvipsnames,svgnames,table]{book}
\usepackage[spanish,es-tabla]{babel}
\usepackage[utf8]{inputenc}
\usepackage[round]{natbib}
\usepackage{graphicx}
\usepackage{multirow}
\usepackage{subfigure}
\usepackage{capt-of}
\usepackage{multirow,array}
\usepackage{float}
\usepackage{vmargin}
\usepackage{etoolbox}
\usepackage[Sonny]{fncychap}
\usepackage{hieroglf}
\usepackage{pdfpages}
\usepackage{color}
\usepackage{subfiles}
\usepackage{amsmath}
\usepackage{ragged2e}
\usepackage[hidelinks]{hyperref}
%\usepackage{minted}
%\usepackage{xcolor}

\setpapersize{A4}
\setmargins{35pt}       % margen izquierdo
{1cm}                  	 % margen superior
{16.9cm}                 % anchura del texto
{22cm}               	 % altura del texto
{14pt}                   % altura de los encabezados
{2cm}                    % espacio entre el texto y los encabezados
{0pt}                    % altura del pie de página
{2cm}                    % espacio entre el texto y el pie de página


\makeatletter	%Tamaño de encabezados
\patchcmd{\@makechapterhead}{\vspace*{50\p@}}{\vspace*{-30\p@}}{}{}
\patchcmd{\@makeschapterhead}{\vspace*{50\p@}}{\vspace*{-30\p@}}{}{}
\patchcmd{\DOTI}{\vskip 50\p@}{\vskip 0\p@}{}{}
\patchcmd{\DOTIS}{\vskip 5\p@}{\vskip 0\p@}{}{}
\makeatother

\title{Manual de usuario del robot Justina}
\author{Laboratorio de Biorobótica}
\begin{document}
\maketitle

\tableofcontents

\chapter{Introducción}

\chapter{Hardware}

\section{Componentes}
Base móvil: 
* Cuatro
* Clemas
* roboclaw
\section{Diagramas esquemáticos}


\chapter{Arquitectura del Software}
\section{ViRbot}
\section{Guía de desarrollo}
\section{Árbol de carpetas}
\section{Instalación}
Comentarios sobre cómo se migraría a otras versiones de ROS y de Ubuntu

\chapter{Nodos de ROS}

\section{Hardware}
Sólo los nodos de esta carpeta interactúan con hardware. Todos los demás nodos sólo usan tópicos y servicios para obtener info del hardware. 
\subsection{Head}

\subsection{Arms}

\subsection{Mobile base}
Publica la tf de base-link a odom
Publica un tópico pose que corresponde a la odometría
\subsection{Descripción del robot: URDF}
head\_pan        head\_node head\_simul\_node

head\_tilt

\section{Navigation}


\section{Vision}
\section{Human-robot interaction}
\subsection{gui}
Sirve para operar todo el robot desde una interfaz gráfica cuyas funciones se detallan en la sección \ref{sec:GUI}.
\section{Interoperation}

\section{Planning}

\section{Paquetes de terceros}
\subsection{hokuy-node}
\subsection{amcl}
\subsection{robot-state-publisher}

\section{Puesta en marcha}
\subsection{Usar al robot con la GUI}
\label{sec:GUI}


\end{document}